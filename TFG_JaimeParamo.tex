% Plantilla TFG desarrollada por Víctor Gayoso Martínez
% U-tad, 2023

\documentclass[oneside,a4paper,12pt]{book} 

\usepackage[spanish, es-nodecimaldot]{babel}
\usepackage[T1]{fontenc}
\usepackage[utf8]{inputenc}
\usepackage{geometry}
\usepackage{amsmath,amsfonts} % Para poder usar \mathbb
\usepackage{makeidx}
\usepackage{url}
\usepackage{graphicx}
\usepackage{color}
\usepackage{caption}
\usepackage{acronym}
\usepackage{hyphenat}
\usepackage{a4wide}
\usepackage[normalsize]{subfigure}
\usepackage{float}
\usepackage{titlesec}
\usepackage[Lenny]{fncychap}
\usepackage{listings} % para poder hacer uso de "listings" propios (p.ej. códigos)
\usepackage{eurosym} % para poder usar el símbolo del euro con \euro {xx}

% *** DATE ***
\usepackage{datetime2} 
\newcommand{\fechaEntrega}{\DTMsetstyle{spanish}\DTMDate{2024-06-10}}

% *** TABLES *** 
\usepackage{booktabs}
\usepackage{colortbl}

% *** HYPERLINKS ***
\usepackage{hyperref}
\hypersetup{
	linkcolor=blueSelf, 
	colorlinks=true, 
	urlcolor=blueSelf,
	citecolor=blueSelf
}

% *** COLOR ***
\usepackage{xcolor}
\definecolor{blueSelf}{RGB}{0,0,255}

% *** MARGINS ***
\usepackage{changepage}

\usepackage{times}

\usepackage{subfigure}
\usepackage[subfigure]{tocloft}

\usepackage{enumitem}

% Para las referencias (es necesario utilizar biber)
\usepackage{csquotes}
\usepackage[style=apa,backend=biber]{biblatex}
\DeclareLanguageMapping{spanish}{spanish-apa}
\addbibresource{Bibliografia/biblio.bib} % Fichero donde se incluyen las referencias

% Para que se rellene con puntos el índice general
\renewcommand\cftchapdotsep{\cftdotsep}
\renewcommand\cftchapleader{\cftdotfill{\cftchapdotsep}}

% Para que el índice no sea clickable
\makeatletter
\let\Hy@linktoc\Hy@linktoc@none
\makeatother

% Para que no divida las palabras
\pretolerance=10000

\newcommand{\bigrule}{\titlerule[0.5mm]} \titleformat{\chapter}[display] 
{\bfseries\Huge} 
{
\filright 
\Large\chaptertitlename\ 
\Large 
\Huge \Large\thechapter} 
{0mm}
{\filright}
[\vspace{0.5mm}
] 
\geometry{a4paper, left=2.25cm, right=2.0cm, top=3cm, bottom=2cm, headsep=1.5cm}

\usepackage{fancyhdr}

\fancypagestyle{plain}{} % Para que se muestre la cabecera y el número de página en la primera página de un capítulo

\pagestyle{fancy}
\fancyhf{}
\fancyhead[L]{\nouppercase\leftmark}
\fancyfoot[C]{\nouppercase\thepage}

\addto\captionsspanish{\def\listtablename{\uppercase{\'{I}}\lowercase{ndice de tablas}}}

% Especifica las reglas de separación que consideres. Algunos ejemplos:
\hyphenation{fuer-tes}
\hyphenation{mul-ti-ca-pa}
\hyphenation{res-pues-ta}
\hyphenation{di-fe-ren-tes}
\hyphenation{de-sa-rro-lla-dos}
\hyphenation{re-pre-sen-tan-do}

% Opciones de distancias con las enumeraciones
\setlist[itemize,1]{leftmargin=0.65cm,labelindent=0.0cm,labelsep=0.2cm}

\setlist[enumerate,1]{leftmargin=0.65cm,labelindent=0.0cm,labelsep=0.2cm}

\renewcommand\labelenumi{\theenumi)}

\usepackage{lipsum}

%\usepackage{indentfirst}        % Sangrado de parrafos al estilo europeo

% Establece la profundidad del índice y la numeración de las páginas
\setcounter{secnumdepth}{2} \setcounter{tocdepth}{1}

\usepackage{longtable} % Para tablas muy largas
\usepackage{multirow} % Para agrupar varias filas en las tablas

\usepackage{wasysym} % Para \RHD en los bullets de segundo nivel

% Para la distancia entre párrafos
\setlength\parskip{0.4em plus 0.1em minus 0.1em}
\setlength\parindent{0pt}

\renewcommand\labelitemi{\raisebox{0.35ex}{\tiny$\bullet$}}
\renewcommand{\labelitemii}{$\RHD$}

% Para cambiar el espaciado entre el número de sección y el título
\makeatletter 
\renewcommand{\@seccntformat}[1]{\csname the#1\endcsname\hspace{1ex}}

% *** SANGRÍA PÁRRAFO ***
\usepackage{indentfirst}
\setlength\parindent{1.25cm}

% *** SUBFIGURAS ***
\usepackage{subcaption}

\makeatother
\begin{document}

% Si se pone antes de \begin{document} no funciona
\renewcommand{\tablename}{\bfseries Tabla} 
\renewcommand{\figurename}{\bfseries Figura}

\baselineskip 1.35\baselineskip

\frontmatter

% PORTADA
\thispagestyle{empty}


\phantom{xxxx}
\vspace{-2.0cm}
\begin{center}
\begin{tabular}{ l p{2.6cm} r }
\includegraphics[width=0.25\textwidth]{./Figuras/LogoUCJC.png} &  & \begin{tabular}{l}\includegraphics[width=0.40\textwidth]{./Figuras/LogoUtad.png} \\[3.4cm] \phantom{x}  \end{tabular}\\ 
\end{tabular}
\end{center}

%\vspace{0.2cm}
\begin{center}

{\Huge {\textbf{Videojuegos y Musicoterapia:}}}

\vspace{0.35cm}
{\Huge {\textbf{Diseño de Experiencias Rítmicas}}}

\vspace{0.25cm}
{\Huge {\textbf{para Mejorar el Bienestar Emocional}}}\\[1.3cm]
\end{center}

\vspace*{\stretch{4}}

\begin{center}
{\Huge {\textbf{Trabajo de Fin de Grado}}}
\end{center}

\vspace*{\stretch{5.3}}

\renewcommand{\arraystretch}{1.5}

\begin{longtable}{l p{13.7cm}}
	{\Large \textbf{Convocatoria: }} & {\Large Extraordinaria}\\[0.3cm]
{\Large \textbf{Alumno/a: }}& {\Large Jaime Páramo Benítez} \\[0.3cm]
	{\Large \textbf{Tutor/a: }}  & {\Large Eva Perandones Serrano} \\[0.3cm]
	{\Large \textbf{Cotutor/a: }}  & {\Large Javier Alegre Landaburu} \\[0.3cm]
	{\Large \textbf{Grado: }} & {\Large Diseño de Productos Interactivos} \\
\end{longtable}

\vspace*{\stretch{4}}

\mainmatter

\pagenumbering{arabic}

% AGRADECIMIENTOS
\newpage {\pagestyle{empty}}
\markboth{Agradecimientos}{Agradecimientos}
\chapter*{Agradecimientos}
\vspace{2cm}
\thispagestyle{empty}


%\phantom{x}

%\vspace{1.0cm}
%\begin{center}
%{\LARGE \textbf{AGRADECIMIENTOS}}
%\end{center}

\vspace{-1.8cm}
Me gustaría agradecer a toda mi familia y a Enrique, sin olvidarnos de Diego Armando Maradona, amante de la nieve.

% ÍNDICE DE CONTENIDOS
\newpage{\pagestyle{empty}}
\setcounter{page}{1}
\tableofcontents

% ÍNDICE DE FIGURAS
\newpage{\pagestyle{empty}}
\addcontentsline{toc}{chapter}{\numberline{}Índice de figuras}
\listoffigures

% ÍNDICE DE TABLAS
\newpage{\pagestyle{empty}}
\addcontentsline{toc}{chapter}{\numberline{}Índice de tablas}
\listoftables

% ABSTRACT
\newpage{\pagestyle{empty}}
\markboth{Abstract}{Abstract}
\chapter*{Abstract}
\addcontentsline{toc}{chapter}{\numberline{}Abstract}
The aim of the ARTEMIS project is to develop an application that serves as a complementary tool in music therapy sessions that aim to help patients move from a state of anxiety to one of calm. To achieve this goal, research has been carried out on the functioning of anxiety, its possible treatments, music therapy therapies and their applications in the relevant psychological field, as well as the design elements of serious health-oriented video games. An application has been developed that encompasses various activities with common goals, using technologies that enable the creation of interactive digital experiences such as Unity, audio management such as FMOD, and collaborative work in repositories such as GitHub. Specifically, the activity that has been developed in this work immerses the patient in a relaxing rhythmic experience in which they will have to find a subjective solution to a musical puzzle. The elaboration of this experience following the therapeutic bases of relaxation and the interactivity of the user-centered digital medium, allows the therapist to adapt to the needs of each patient, and even to the different needs of the same patient at different stages of therapy.

\textit{Keywords:} video games, music therapy, rhythm, mental health, anxiety, ARTEMIS

% GLOSARIO
\newpage{\pagestyle{empty}}
\markboth{Glosario}{Glosario}
\chapter*{Glosario}
\addcontentsline{toc}{chapter}{\numberline{}Glosario}

\renewcommand{\arraystretch}{1.5}



\begin{longtable}{l p{13.7cm}}
	
\textbf{DAW} & Digital Audio Workstation \\
\textbf{IDE} & Integrated Development Environment \\
\textbf{LTS} & Long-Term Support \\
\textbf{NPC} & Non-Player Character \\
\textbf{SIE} & Sony Interactive Entertainment \\
\textbf{TAG} & Trastorno de Ansiedad Generalizada \\
\textbf{TFG} & Trabajo de Fin de Grado \\
\textbf{TOC} & Trastorno Obsesivo-Compulsivo \\

\end{longtable}


% CAPÍTULO 1 - INTRODUCCIÓN
\newpage{\pagestyle{empty}}
\chapter{Introducción}  
Esta investigación está vinculada al proyecto ARTEMIS, un proyecto de investigación enfocado en el desarrollo de experiencias digitales interactivas basadas en musicoterapia. Estas experiencias se agrupan en una aplicación diseñada para funcionar como un puente entre el médico y el paciente en terapias psicológicas. El objetivo es ayudar a los pacientes a transitar de emociones con connotaciones negativas hacia emociones con connotaciones positivas.

La investigación se ha centrado en la ansiedad infantil y su tratamiento mediante la musicoterapia. Sin embargo, el proyecto ARTEMIS aspira a ser escalable para diferentes grupos de edad y emociones. Inicialmente, se consideró enfocar la investigación en grupos vulnerables como niños y ancianos. No obstante, se encontró que los niños interactúan más fácilmente con entornos digitales, por lo que se decidió que ese sería el punto de partida para ARTEMIS.

Se ha desarrollado e integrado una experiencia en la aplicación con el objetivo de facilitar la transición de un estado de ansiedad a la calma. A lo largo del documento, se explicará cómo se logra esta transición a través de una ordenación rítmica que pasa del caos al orden subjetivo, mediante la interacción del paciente con las instrucciones proporcionadas por el terapeuta.

\begin{figure} [h!]
	\centering
	\includegraphics[width=0.9\linewidth]{Figuras/Introduccion/1_LogoArtemis}
	\caption{Logotipo del proyecto ARTEMIS.}
	\label{fig:logoArtemis}
\end{figure}

\section{Justificación y contexto}

El proyecto ARTEMIS se crea con el propósito de proporcionar soporte digital a las terapias psicológicas que tienen como objetivo utilizar la musicoterapia como medio para facilitar la transición entre estados emocionales. Su función radica en ser una herramienta complementaria que los terapeutas pueden implementar junto a las terapias más tradicionales, adaptándola a las necesidades de cada paciente. La aplicación, que conglomera las distintas experiencias, no debe ser utilizada solo por el paciente. Tiene un enfoque de doble usuario, donde ambas partes tienen su rol definido en la terapia. El paciente interactúa con la aplicación siguiendo las indicaciones del terapeuta, quién recogerá información relevante tanto antes como después de la interacción. De este modo, puede recibir feedback a tiempo real y redirigir las sesiones terapéuticas si fuera necesario. Además, el terapeuta tiene acceso a un registro que le permite observar la progresión del paciente. Esto le ayuda a identificar patrones y a priorizar los elementos que funcionan mejor en cada caso específico.

La investigación busca aportar una síntesis entre el diseño de productos interactivos y la musicoterapia, fundamentada por estudios previos sobre el uso de la música en el tratamiento de trastornos emocionales, en particular, la ansiedad. Esta emoción fue seleccionada para su estudio debido a su prevalencia en la sociedad y a su presencia a lo largo de la vida de una persona, teniendo un gran impacto, especialmente en la etapa de la niñez y adolescencia. Un estudio conducido por \citeauthor{MO:2012} (\citeyear{MO:2012}) indica que, de una muestra de aproximadamente 2500 niños y adolescentes entre 8 y 17 años (51\% de sexo femenino), de diversas nacionalidades (aunque predominantemente española) y diferentes situaciones socioeconómicas, el 26,41\% presentó puntuaciones altas en algún tipo de ansiedad. Aunque se hable de esta emoción como una única, \citeauthor{CAGD:2010} (\citeyear{CAGD:2010}) explica que se caracteriza por ser un conjunto de diferentes tipos que surgen como reacción a eventos o situaciones. Estos tipos están condicionados por las circunstancias que rodean al individuo y los recursos que tiene, conocidos como estrategias de afrontamiento. Entre las reacciones a la ansiedad se incluyen la incertidumbre, la impotencia y la activación fisiológica, fuertemente relacionadas con síntomas corporales de tensión y preocupación acerca del futuro. Los síntomas pueden intensificarse significativamente cuando se combina con la depresión, llegando incluso a la somatización. Como explica \citeauthor{CJI:2017} (\citeyear{CJI:2017}), la ansiedad puede hacer que la mente enferme al cuerpo. Aunque aún no podemos explicar el origen de las causas, el dolor y sufrimiento se manifiestan de forma real. 

La música y la salud están estrechamente relacionadas. La combinación adecuada de elementos musicales como tonos, ritmos y armonías puede impactar positivamente las condiciones físicas, fisiológicas y psicológicas de las personas. Aquí es donde entra la musicoterapia, un enlace entre el médico y el paciente que busca cumplir un objetivo terapéutico específico a través de la interacción con medios musicales. En nuestro caso de estudio, el objetivo es reducir la ansiedad. \citeauthor{KTN:2011} (\citeyear{KTN:2011}) explica que los patrones de las ondas cerebrales cambian según el estado de ánimo del paciente que, combinados con ciertos tipos de música, pueden generar un equilibrio que conduce a la relajación. En los últimos años, se han empezado a implementar nuevas estrategias dentro de las sesiones de musicoterapia, incluyendo el uso de videojuegos como medio interactivo.

\section{Motivación}

La elección de la temática para este Trabajo de Fin de Grado se debe a la fusión de dos de mis mayores pasiones: los videojuegos y la música. Desde una edad temprana, ambos han jugado un papel crucial en mi vida, no solo como formas de entretenimiento, sino como medios de expresión y canales que me han servido para desarrollar la creatividad. La influencia personal de estas dos áreas se puede apreciar tanto desde la perspectiva del desarrollador o intérprete, como desde el punto de vista del jugador u oyente. El diseño y desarrollo de experiencias interactivas para la musicoterapia es el punto de convergencia de estas dos pasiones, donde se agrega el elemento psicológico, un tema de especial interés para mí. 

A través de la investigación y desarrollo de esta experiencia interactiva, pretendo potenciar mis conocimientos en los campos específicos a los que me gustaría dedicarme profesionalmente en la industria del videojuego. Este proyecto no solo me permitirá aplicar las habilidades técnicas y creativas que he ido desarrollando a lo largo de mi carrera educativa, sino que también me proporcionará una comprensión más profunda de cómo la música puede ser utilizada de manera terapéutica dentro de estos entornos interactivos. Además, me ayudará a comprender cómo, recopilar y analizar datos sobre la respuesta del usuario y los resultados terapéuticos, permite adaptar las sesiones terapéuticas al paciente a tiempo real, lo que fortalecerá mis habilidades de evaluación.

Mi gran curiosidad por la mente humana, que considero una pieza clave en la máquina perfecta que es el cuerpo humano, me impulsa a explorar cómo estas aplicaciones interactivas pueden influir en el cerebro y el comportamiento. Este proceso también me aportará información que alimente esta curiosidad en los campos psicológicos. Al investigar cómo las experiencias interactivas basadas en la música pueden influir psíquica y fisiológicamente, espero descubrir nuevas maneras en las que la tecnología puede ser utilizada para mejorar la salud mental y el bienestar general. Esta exploración que entrelaza disciplinas no solo enriquecerá mi formación académica y profesional, sino que también contribuirá a un campo de estudio en constante evolución, con el potencial de tener un impacto positivo y duradero en la vida de las personas.

\section{Planteamiento del problema}

La musicoterapia, tal como la conocemos hoy, se originó en la segunda mitad del siglo XIX. Sin embargo, las civilizaciones antiguas ya utilizaban la música como un medio divino para apaciguar a los dioses, vinculando la enfermedad con la maldad y la ofensa a estos (\citeauthor{JIPS:2001}, \citeyear{JIPS:2001}). Está claro que la musicoterapia ha sido una parte esencial de la vida humana durante milenios. A lo largo de la historia, variadas culturas han reconocido el poder curativo de la música, desde los rituales chamánicos de las tribus indígenas hasta las complejas composiciones de la Grecia clásica, donde filósofos como Pitágoras exploraron los efectos de los sonidos en el alma y el cuerpo.

Sin embargo, debido a la relativa reciente aparición de los medios digitales en comparación con la larga historia de la musicoterapia, el desarrollo de terapias interactivas digitales que utilizan la música como elemento central está en proceso de crecimiento. Esto implica que, a pesar del incremento de uso de este formato en diversas áreas científicas y el constante desarrollo de este tipo de experiencias, aún no ha logrado establecerse completamente en el área psicológica.

Actualmente, existen pocas investigaciones recientes sobre el uso de medios interactivos electrónicos en sesiones de musicoterapia, a pesar de su creciente presencia en otros sectores científicos, incluyendo otras áreas de la psicología. Esta brecha en la investigación representa una oportunidad significativa para explorar y desarrollar nuevas metodologías que integren eficientemente la tecnología digital con prácticas terapéuticas tradicionales. Las experiencias digitales interactivas tienen el potencial de ofrecer entornos controlados y adaptativos que pueden personalizarse para las necesidades individuales de cada uno de los pacientes, algo que puede ser más difícil de lograr con los métodos tradicionales.

El límite del diseño de videojuegos se encuentra en la imaginación del propio creador. Al combinar terapias tradicionales con medios digitales, se abre un rango más amplio de posibilidades terapéuticas. Esta combinación compensará las limitaciones físicas con las ventajas digitales, aprovechando lo mejor de ambos mundos. Es importante destacar que el formato digital no tiene como objetivo reemplazar las terapias tradicionales, sino funcionar como una herramienta complementaria que el terapeuta puede utilizar en sus sesiones, adaptándolas a las necesidades de cada paciente. Además, la interacción digital introduce nuevas formas de involucrar a los pacientes, manteniendo la interacción física del médico con el paciente, permitiendo medir su progreso de manera objetiva, y lo que es más importante, en tiempo real.

\section{Objetivos del trabajo}

Teniendo en cuenta las motivaciones y problemas que se han planteado en los epígrafes anteriores, los objetivos que el trabajo pretende alcanzar son los siguientes:

\subsection{Objetivos generales}

\begin{itemize}
	\item Investigar cómo las experiencias interactivas digitales pueden complementar y mejorar las prácticas tradicionales dentro del área psicológica de la musicoterapia.
	\item Desarrollar una aplicación con enfoque de juego serio que incorpore elementos de musicoterapia y que pueda ser utilizado en sesiones terapéuticas, que tengan como objetivo mejorar el estado emocional con respecto a la ansiedad.
\end{itemize}

\subsection{Objetivos específicos}

\begin{itemize}
	\item Evaluar el impacto de estas experiencias basadas en música sobre el bienestar emocional y físico de los pacientes.
	\item Estudiar cómo los usuarios interactúan con el videojuego, identificando patrones de uso y preferencias que puedan mejorar futuras iteraciones en el diseño del juego.
\end{itemize}

% CAPÍTULO 2 - ESTADO DE LA CUESTIÓN
\chapter{Estado de la cuestión}  
%\addcontentsline{toc}{chapter}{\numberline{}Estado de la cuestión}    
A lo largo del estado de la cuestión, será necesario abordar los campos relevantes de la psicología, la música y los videojuegos para fundamentar el desarrollo de la aplicación con investigaciones pertinentes. A continuación, se detallan los campos de interés que contribuirán a alcanzar los objetivos de la investigación.

\section{Marco teórico del trabajo}

\subsection{Ritmo, creación musical y emociones}

\subsubsection{Definición de ritmo y efectos terapéuticos}

El ritmo es un elemento intrínseco de la vida humana. Se manifiesta en la mayoría de las formas de arte, siendo de gran importancia en la música, la poesía y la danza. La definición de ritmo según \citeauthor{CONCEPTO:2021} (\citeyear{CONCEPTO:2021}) es la siguiente:

\begin{adjustwidth}{100pt}{0pt}
	''\textit{Se denomina ritmo a todo movimiento regular y recurrente, marcado por una serie de eventos opuestos o diferentes que se suceden en el tiempo. Dicho en otras palabras, el ritmo es un fluir del movimiento de naturaleza visual o sonora, cuyo orden interno puede percibirse e incluso reproducirse.}''
\end{adjustwidth}

La percepción del ritmo es totalmente subjetiva, pero para comprenderlo es esencial encontrar la clave que lo hace común para todos los individuos de la sociedad. Un ritmo siempre será un ritmo si los eventos que ocurren en el tiempo definen un patrón y un margen de repetición. Existen distintos tipos de ritmo en varias áreas artísticas. Sin embargo, nos centraremos en el ritmo musical, que es el más relevante para el campo psicológico de la musicoterapia. El ritmo musical se compone de varios elementos que indican su velocidad, intensidad y duración. El pulso, el tempo y la métrica son los elementos principales que definen un ritmo musical. Abordaremos individualmente cada uno de estos términos, indagando en profundidad que función tienen dentro de la música.

\begin{itemize}
	\item \textbf{Pulso:} es considerado como el latido del corazón de la música. Por definición, el pulso consiste en una serie de pulsaciones que se repiten de manera constante en una pieza musical, vinculadas directamente al movimiento del pie al escuchar una canción. El pulso puede tener diferentes velocidades, ya sean más rápidas o más lentas, pero siempre es constante a lo largo de la pieza musical. Sin embargo, existen excepciones si el compositor de la música decide acelerar o disminuir la velocidad intencionalmente (\cite{VIOLÍNZN:2024}).
	\item \textbf{Tempo:} proveniente del italiano ''tiempo'', el tempo se refiere a la velocidad de una composición. Metafóricamente, es similar a un reloj, que nos indica cuándo realizar ciertas acciones. El tempo permite a los músicos conocer el momento preciso en el que tocar cada sección de una pieza musical. Cuando los pulsos están distanciados equitativamente en el tiempo, la unidad de medida del tempo son BPM (\cite{MASTEREDBLOGS:2021}). El tempo puede ser constante, como en algunas canciones contemporáneas, o variable. Es común encontrar movimientos con distintos tempos en las piezas clásicas.
	\item \textbf{Métrica:} en el punto medio entre los dos términos anteriores, encontramos la métrica. Esta combina ambos elementos y define la estructura de una composición musical. La métrica se refiere a cómo se organizan los pulsos y el tempo en una pieza musical, agrupando los pulsos en unidades llamadas compases. En cada compás, algunos pulsos sonarán más fuertes y otros más suaves, lo cual se conoce como acentuación. La combinación de todos estos elementos proporciona un sentido rítmico a la música (\cite{COMPOMUSICAL:2024}). La métrica incorpora todas las figuras necesarias (\autoref{fig:MusicalFigures}) para visualizar la música en una partitura. La música se percibirá de manera diferente dependiendo de la altura, duración y tipo de figura que contengan la combinación de compases que conforman la pieza.
\end{itemize}

\begin{center}
	\textbf{Fuente:} \citeauthor{VIOLÍNZN:2024} (\citeyear{VIOLÍNZN:2024}).
	\vspace{-18pt}
\end{center}

\begin{figure}[h!]
	\centering
	\includegraphics[width=0.4\linewidth]{Figuras/Estado/FigurasMusicales.jpg}
	\caption{Figuras musicales, incluyendo detalles sobre su nombre, duración, y la figura de silencio correspondiente.}
	\label{fig:MusicalFigures}
\end{figure}

El ritmo tiene potencial terapéutico, ya que puede generar efectos positivos en el cuerpo y la mente. Según \citeauthor{JARAH:1977} (\citeyear{JARAH:1977}), el núcleo de un ritmo se define por dos fases: movimiento y ausencia de movimiento. A nivel fisiológico, estas dos fases están directamente relacionadas con la tensión y relajación. Aunque medir la música en función del tiempo puede ser deshumanizante, debemos hacerlo para facilitar la adaptación del intérprete a la pieza. La medición del tiempo en la música se hace de manera cíclica. Por eso, el ritmo es un elemento cíclico, ya que la mayoría de nuestras acciones dependientes del tiempo tienen esta propiedad. La respiración, una acción cíclica definida por la inhalación y la exhalación (Arsis y Tesis en términos musicales\footnote{Arsis y Tesis son las partes más fuerte y más débil de un compás musical, respectivamente. Se utilizan tanto en música como en poesía. Se definen en función de la dirección de la línea melódica y la duración de la última nota del compás con respecto al tono semifuerte (\cite{CASTELLANOS:2009})} (\cite{CASTELLANOS:2009})), es la acción humana cíclica más relevante. 

Entre los efectos terapéuticos que incluye el ritmo podemos encontrar la mejora de la coordinación motora, que ayuda a incrementar la sincronización de los pacientes, y la regulación emocional, promoviendo la calma con ritmos lentos o estimulando la energía con ritmos rápidos. Un ejemplo clínico del uso del ritmo se puede observar en un estudio realizado por \citeauthor{BENSIMON:2008} (\citeyear{BENSIMON:2008}), que muestra cómo los soldados de guerra con trastorno de estrés postraumático (TEPT), que presentan síntomas de soledad y aislamiento social, entre otros, se benefician de terapias de grupo. El ritmo es el elemento principal para su mejoría.

\subsubsection{Proceso de composición, formas y tipos}

Según \citeauthor{RAE:2024} (\citeyear{RAE:2024}), el término ''creación'' se refiere a la acción y efecto de crear. El acto de crear es muy amplio, ya que se puede adaptar a cualquier circunstancia en la que se genere algo de la nada. En nuestro contexto de estudio, la creación o composición musical consiste en organizar sonidos y silencios para transmitir emociones y expresar ideas (\cite{MMARTIN:2024}). Por lo tanto, la creación e invención musical es el punto culminante de la actividad creativa musical (\cite{SUBIRATS:2004}). La selección del ritmo, la melodía, la armonía y la textura son algunos de los elementos fundamentales para la composición musical. Para entender el proceso de composición, debemos considerar los niveles de creatividad de la mente humana. Según \citeauthor{SUBIRATS:2004} (\citeyear{SUBIRATS:2004}), existen cinco niveles, que se describen de la siguiente manera:

\begin{itemize}
	\item \textbf{Creatividad expresiva:} es la forma más básica de creación. Se basa en el descubrimiento de nuevas formas de expresar sentimientos, permitiendo al individuo identificarse con su propia obra y mejorar su comunicación con los demás. Además, es la forma de creación más presente en cualquier campo de expresividad como la música, la poesía o la danza. 
	\item \textbf{Creatividad productiva:} este tipo de creatividad se centra en la aplicación de técnicas y estrategias que ayudan al individuo a alcanzar un objetivo previamente fijado. Está orientada hacia la eficiencia de productividad, especialmente en situaciones que requieren la producción a gran escala.
	\item \textbf{Creatividad incentiva:} entre la expresividad (espontaneidad) y la producción (presupuestos), surge la creatividad incentiva, que se caracteriza por una mayor capacidad para descubrir nuevas realidades. En este caso, el ambiente juega un papel determinante.
	\item \textbf{Creatividad innovadora:} permite al individuo lograr resultados únicos a través de la transformación del medio. Supone una mayor flexibilidad de invención y mejora las capacidades creadoras del individuo.
	\item \textbf{Creatividad emergente:} esta cualidad define al genio que posee un gran talento a nivel creativo. No se trata de producir basándose en modificaciones de productos antiguos, sino que la creación de ideas se fundamenta en principios completamente nuevos.
\end{itemize}

La composición musical reúne todos estos niveles de creatividad. Existen dos formas de crear en este ámbito: la improvisación libre y la composición guiada. Cuando nos referimos a la improvisación libre, no hablamos de cualquier tipo de improvisación. Comúnmente, se entiende la improvisación como el acto de "tocar cualquier cosa". \citeauthor{OLMEDO:2014} (\citeyear{OLMEDO:2014}) define este tipo de improvisación con la frase: "Hay pacientes que improvisan pero no le doy estatuto de improvisación porque tocan de un modo maquinal y no escuchan". Sin embargo, la improvisación libre requiere un nivel de profundidad adicional del individuo. Debe entender las bases fundamentales de lo que está haciendo y por qué lo está haciendo, sin desviarse de las funciones de la improvisación. Además, la creación musical puede desarrollarse a través de varios formatos, dependiendo del instrumento que genera la música. La voz, objetos cotidianos o instrumentos musicales son los principales medios para la creación o composición musical.

\subsubsection{Relación entre la música y las emociones}

Las emociones y la música están estrechamente relacionadas. Aunque los gustos musicales son totalmente subjetivos, ciertos tipos de música provocan, en niveles distintos, las mismas emociones en todos los seres humanos. Normalmente, el cerebro funciona por regiones, activando ciertas áreas según la acción que se esté realizando. Sin embargo, la escucha activa de la música no parece activar ninguna zona específica, sino que se distribuye por todo el cerebro. Por lo tanto, cuando las personas reaccionan al estímulo de la música, experimentan diferentes sensaciones en el cuerpo. Esto se debe a que la música provoca un cambio tanto fisiológico como psicológico, una reacción conocida como biomúsica (\cite{MOSQUERA:2013}).

Un estudio realizado por \citeauthor{MOSQUERA:2013} (\citeyear{MOSQUERA:2013}), observó que al escuchar música agradable, se pueden activar ciertas sustancias químicas en el Sistema Nervioso Central. Esto estimula la producción de neurotransmisores como la dopamina, las endorfinas y la oxitocina, y se experimenta un estado que favorece la alegría y el optimismo en general. Sin embargo, la respuesta emocional puede variar entre individuos, ya que se refleja en función de las experiencias personales y aprendizajes previos de cada uno.

La expectativa musical desempeña un papel crucial en la gestión de emociones, ya que puede generar satisfacción o frustración, dependiendo de cada individuo y sus previsiones. En la expectativa musical, los oyentes formulan una serie de hipótesis sobre cómo continuará una pieza musical desde que suena la primera nota. Estas expectativas se fundamentan en una serie de teorías que \citeauthor{SENABRE:2019} (\citeyear{SENABRE:2019}) describe de la siguiente manera:

\begin{itemize}
	\item \textbf{Teoría sobre las Expectativas de Leonard B. Meyer:} la respuesta emocional del oyente puede interpretarse como resultado de las desviaciones en los eventos percibidos respecto a las normas estilísticas. El incumplimiento de la norma puede ser un estímulo que provoca la respuesta emocional del oyente.
	
	\begin{adjustwidth}{100pt}{0pt}
		\textit{''El disfrute provocado por la escucha musical surge de la percepción del juego del artista con formas y convenciones que están arraigadas como hábitos de percepción tanto en el artista como en el público. (…) No consiste en un interés intelectual en detectar semejanzas y diferencias, sino en el inmediato goce estético que resulta de la aparición y la suspensión o el cumplimiento de expectativas que son producto de muchos encuentros previos con obras de arte.''}
		\begin{flushright}
			\vspace{-10px}
			(\cite{MEYER:1956})
		\end{flushright}
	\end{adjustwidth}
	
	\item \textbf{Modelo de Implicación-Realización de Narmour:} el trabajo que Meyer comenzó fue continuado por Eugene Narmour, lo que permitió la comprobación empírica de algunos de sus aspectos. Este modelo sugiere que durante la escucha, se generan dos fuentes de expectativas asociadas a los dos sistemas de implicación melódica. De manera directa se hace alusión a los dos tipos de procesos que intervienen en la percepción: uno ascendente (procesos guiados por datos) y uno descendente (procesos guiados conceptualmente).
	\item \textbf{Modelo de Alfabetos de Deustch y Feroe:} Diana Deustch y John Feroe propusieron una teoría para formalizar cómo los oyentes representamos temporalmente las secuencias de alturas en la música tonal occidental, y cómo utilizamos esas representaciones para prever su continuación. Utilizaron un sistema de alfabeto para definir los distintos niveles de comprensión de la música. En la base de su alfabeto se encuentra la escala cromática.
	\item \textbf{Fuerzas Musicales de Steve Larson:} \citeauthor{LARSON:2002} (\citeyear{LARSON:2002}) propone una teoría en la que tres fuerzas, similares a las que inciden en los movimientos en el espacio físico, condicionan el comportamiento de los patrones musicales. Según la teoría de las fuerzas musicales, los oyentes anticipamos que la gravedad, el magnetismo y la inercia influirán en la continuidad de las secuencias escuchadas, generando la expectativa de una resolución completa.
	\item \textbf{Teoría de la Atención Dinámica de Mari R. Jones:} a diferencia de las teorías anteriores, la teoría de la Atención Dinámica propone un enfoque que pueda modelar la percepción y nuestras expectativas poniendo especial énfasis en el carácter dinámico de esta capacidad humana.
	
	\begin{adjustwidth}{100pt}{0pt}
		\textit{''La estructura del mundo es producto de nuestra capacidad para detectar cambios a lo largo de las dimensiones físicas (es decir, valores contrastantes) y de nuestra tendencia a representar estos cambios como relaciones a lo largo de dimensiones subjetivas.''}
		\begin{flushright}
			\vspace{-10px}
			(\cite{JONES:1976})
		\end{flushright}
	\end{adjustwidth}
\end{itemize}

La combinación de las experiencias personales y el aprendizaje previo de cada individuo conforman sus distintas reacciones al escuchar música y sus respectivas evocaciones emocionales. Es muy importante comprender a nivel de educación musical estos aspectos para poder gestionar las emociones de manera efectiva.

\subsection{Musicoterapia y ansiedad}

\subsubsection{Definición de ansiedad}

La ansiedad es una emoción que todos experimentamos en menor o mayor medida en algunos momentos de nuestra vida. Es una respuesta natural del cuerpo ante situaciones de peligro o estrés y puede manifestarse de varias formas. La ansiedad se define generalmente como un sentimiento de miedo, preocupación o malestar relacionado con eventos futuros inciertos. Fisiológicamente, la ansiedad activa el sistema nervioso simpático, desencadenando respuestas corporales como aumento del ritmo cardíaco, sudoración y tensión muscular (\cite{APA:2013}). Emocionalmente, puede provocar sensaciones de temor, y cognitivamente, puede conducir a pensamientos intrusivos acerca de posibles peligros o fracasos.

Según la American Psychological Association (\citeyear{APA:2020}), la ansiedad es distinta al miedo, ya que esta última es una respuesta a una amenaza inmediata, mientras que la ansiedad implica la anticipación a una amenaza futura. La ansiedad es un estado emocional que prepara a la persona para enfrentar potenciales peligros. Esta respuesta puede ser adaptativa cuando se presenta en niveles moderados, ya que puede mejorar el rendimiento y la atención. Sin embargo, si la ansiedad es excesiva y persistente, puede interferir de manera significativa en la vida diaria de una persona. En este último caso, sería crucial consultar a un especialista para identificar el grado de ansiedad que sufre la persona y tratarla de manera adecuada.

Los síntomas de la ansiedad se pueden clasificar en tres dimensiones: físicas, emocionales y cognitivas. Los síntomas físicos incluyen palpitaciones, sudoración, temblores, tensión muscular, dificultad para respirar, mareos, fatiga y problemas gastrointestinales. Son el resultado de la activación del sistema nervioso simpático, que prepara al cuerpo para la lucha o huida en situaciones percibidas como peligrosas (\cite{CRASKE:2016}). A nivel emocional, las personas con ansiedad a menudo experimentan una sensación constante de miedo o preocupación. Es posible que se sientan inquietos, irritables y que perciban una pérdida de control sobre sus emociones (\cite{BARLOW:2002}). En relación a los síntomas cognitivos, la ansiedad puede alterar la forma en que las personas piensan, provocando pensamientos intrusivos, preocupaciones excesivas y dificultades para concentrarse. Estos síntomas cognitivos pueden perpetuar el ciclo de ansiedad. La preocupación constante sobre posibles peligros puede intensificar los síntomas físicos y emocionales (\cite{CLARK:2011}). Todos estos síntomas pueden presentarse simultáneamente, complicando la vida cotidiana de quienes padecen ansiedad. Los niveles de ansiedad están directamente relacionados con la tolerancia de la persona y las circunstancias que rodean la aparición de estos síntomas.

\subsubsection{Tipos de trastornos de ansiedad}
Existen varios tipos de trastornos de ansiedad, cada uno con sus características y criterios diagnósticos propios. Los trastornos de ansiedad más comunes incluyen:

\begin{itemize}
	\item \textbf{Trastorno de Ansiedad Generalizada (TAG):} se caracteriza por una preocupación excesiva e incontrolable en relación con diversas actividades o eventos. Esto se acompaña de síntomas físicos como tensión muscular y problemas de sueño. Las personas con este trastorno suelen anticipar desastres y están constantemente preocupadas por su salud, dinero, familia o trabajo (\cite{APA:2013}).
	\item \textbf{Trastorno de pánico:} este trastorno se caracteriza por ataques de pánico recurrentes e inesperados. Estos son episodios de intenso miedo acompañados de síntomas físicos severos, que incluyen palpitaciones, sudoración, temblores y sensación de asfixia. Las personas con trastorno de pánico a menudo viven con el temor constante de sufrir más ataques. Esto puede llevarlas a evitar situaciones o lugares donde han experimentado ataques anteriormente (\cite{KESSLER:2005}).
	\item \textbf{Trastorno de ansiedad social:} también conocido como fobia social, este trastorno implica un temor intenso y persistente a ser juzgado, avergonzado o humillado en situaciones sociales o de desempeño. Las personas con trastorno de ansiedad social pueden evitar las interacciones sociales o soportarlas con gran angustia (\cite{STEIN:2008}).
	\item \textbf{Trastorno Obsesivo-Compulsivo (TOC):} aunque el TOC se clasifica de manera independiente en el DSM-5\footnote{El DSM-5, actualizado en 2013, es el Manual Diagnóstico y Estadístico de los Trastornos Mentales. Su objetivo es asistir a los profesionales de la salud en el diagnóstico de los trastornos mentales de los pacientes.}, está estrechamente relacionado con los trastornos de ansiedad. Este trastorno se caracteriza por pensamientos obsesivos intrusivos y comportamientos compulsivos repetitivos, los cuales están destinados a reducir la ansiedad causada por estas obsesiones (\cite{APA:2013}).
	\item \textbf{Trastorno de Estrés Postraumático (TEPT):} este trastorno se desarrolla después de que una persona ha sido expuesta a un evento traumático. Los síntomas incluyen la experimentación del trauma a través de flashbacks y pesadillas, la evitación de los recordatorios del evento, y un aumento de la excitabilidad o reactividad (\cite{YEHUDA:2002}).
	\item \textbf{Trastorno de ansiedad por separación:} es más frecuente en niños, pero también puede afectar a adultos. Se caracteriza por un miedo excesivo a la separación de las personas a las que el individuo está apegado. Los síntomas pueden incluir una angustia extrema al anticipar o experimentar la separación, así como preocupaciones constantes de que algo malo suceda a la persona a la que está apegado (\cite{APA:2013}).
	\item \textbf{Fobias específicas:} las fobias son miedos intensos e irracionales hacia objetos o situaciones específicas, como por ejemplo, las alturas, los animales o volar. Estos miedos exagerados pueden provocar una evitación extrema de los estímulos que los causan e interferir significativamente en la vida diaria de una persona (\cite{APA:2013}).
\end{itemize}

A pesar de que los trastornos de ansiedad representan el mayor problema de salud mental en los EE. UU, la mayoría de las personas nunca buscan ayuda médica. Estos muestran una etiología compleja, donde se reconoce el componente genético y los factores estresantes debido a los acontecimientos de la vida. Para su diagnóstico, los médicos y psiquiatras se basan en los criterios clínicos establecidos por el Manual Diagnóstico
y Estadístico de los Trastornos Mentales (DSM-5) (\cite{DELGADO:2021}).

\subsubsection{Terapias tradicionales para la ansiedad}

Para tratar la ansiedad se emplean diversas terapias tradicionales, las cuales han demostrado ser efectivas para las afecciones físicas y psicológicas de los pacientes. Estas se pueden clasificar en terapias psicológicas y tratamientos farmacológicos. Para decidir qué tratamiento utilizar, es necesario realizar una evaluación previa para identificar el tipo de trastorno que tiene el paciente. Aunque ambas opciones se utilizan para tratar la ansiedad, las terapias se consideran tratamientos de primera línea. Estas se combinan para modular los patrones de pensamiento (\cite{DELGADO:2021}).

Las terapias psicológicas son intervenciones respaldadas por evidencia que se enfocan en cambiar los patrones de pensamiento y comportamiento que contribuyen a la ansiedad. Entre las más efectivas están la terapia cognitivo-conductual (TCC), la terapia de exposición, y la terapia de aceptación y compromiso (ACT). Estas terapias no sólo funcionan y se utilizan individualmente, sino que también ofrecen la posibilidad de ser utilizadas de manera complementaria. De esta forma, el terapeuta puede aprovechar las virtudes de cada una para obtener los resultados requeridos en los objetivos terapéuticos.

La TCC es ampliamente reconocida como una de las terapias más efectivas para tratar la ansiedad. Esta se centra en identificar y transformar patrones de pensamiento negativos y comportamientos disfuncionales que alimentan la ansiedad. Los pacientes aprenden técnicas de afrontamiento y estrategias de relajación para manejar su ansiedad de manera más efectiva (\cite{HOFMANN:2012}).

La terapia de exposición, como sugiere su nombre, implica exponer al paciente gradualmente a los estímulos que le causan 	ansiedad en un entorno controlado. El objetivo es disminuir la sensibilidad del paciente, reduciendo su respuesta de ansiedad. La exposición puede ser en vivo, donde el paciente enfrenta directamente la situación temida, o imaginaria, donde visualiza la situación (\cite{CRASKE:2008}). Esta terapia es particularmente eficaz en el tratamiento de fobias específicas, trastorno de pánico y TEPT (\cite{POWERS:2010}).

En cuanto a la terapia ACT, esta se enfoca en aceptar pensamientos y sentimientos difíciles en lugar de resistirse a ellos. Además, promueve el compromiso con acciones que estén en línea con los valores personales del paciente. Esta terapia ayuda a los pacientes a llevar una vida significativa a pesar de la ansiedad, fomentando la flexibilidad psicológica (\cite{HAYES:2012}).

En relación con los tratamientos farmacológicos, estos ayudan a manejar los síntomas de la ansiedad. Los antidepresivos, particularmente los inhibidores selectivos de la recaptación de serotonina (ISRS) y los inhibidores de la recaptación de serotonina y norepinefrina (IRSN), son comúnmente usados para tratar la ansiedad. Estos medicamentos incrementan los niveles de neurotransmisores en el cerebro, lo cual mejora el estado de ánimo y reduce la ansiedad (\cite{BALDWIN:2014}). Por otro lado, los ansiolíticos como las benzodiazepinas proporcionan alivio rápido de los síntomas de ansiedad. No obstante, debido a su potencial de adicción y abuso, su uso generalmente se limita a corto plazo (\cite{FOND:2023}). Los betabloqueantes, como el propranolol, se utilizan para tratar los síntomas físicos de la ansiedad, como taquicardia y temblores. Estos medicamentos bloquean los efectos de la adrenalina, ayudando a reducir los síntomas físicos (\cite{STEENEN:2016}).

\subsubsection{Definición de musicoterapia}

La World Federation of Music Therapy (WFMT) define la musicoterapia de la siguiente manera; citamos literalmente: 

\begin{adjustwidth}{100pt}{0pt}
\textit{''La musicoterapia se refiere al uso profesional de la música y sus elementos como una intervención en entornos médicos, educativos y cotidianos con individuos, grupos, familias o comunidades que buscan optimizar su calidad de vida y mejorar su salud física, social, comunicativa, emocional, intelectual y espiritual. La musicoterapia se utiliza para satisfacer las necesidades físicas, emocionales, cognitivas y sociales de los pacientes}.'' \\
\begin{flushright}
	\vspace{-30px}
	(\cite{WFMT:2024})
\end{flushright}
\end{adjustwidth}

La música se ha utilizado como herramienta terapéutica durante muchos años. Como medio de expresión no verbal, la música facilita la comunicación y la exteriorización de sentimientos, permitiendo a las personas explorar o reexplorar su interior y compartirlo con los demás. El objetivo de usar música es facilitar la expresión emocional del individuo y su desarrollo comunicativo (\cite{TRESIERRA:2005}). Es importante incorporar la música en la educación emocional, especialmente en los niños. Debemos inculcar la importancia de la música como una herramienta en sus vidas que les ayude a gestionar sus emociones. Según \citeauthor{POCH:2001} (\citeyear{POCH:2001}), la música no debe considerarse superflua, ya que ofrece una serie de aportaciones más allá de ser simplemente un pasatiempo. La música tiene un impacto inmediato en los seres humanos, afectando aspectos biológicos, físicos, neurológicos, psicológicos, sociales y espirituales. Además, la música es un patrón autocurativo que la humanidad siempre ha utilizado. Por ejemplo, para aliviar tensiones, cubrir carencias afectivas o expresar sentimientos de alegría a través de la danza, el dolor de una muerte o el amor en canciones románticas. La música acompaña a una persona en todos los momentos esenciales de su vida. Esta presencia se ve reforzada por la cultura, que ha creado canciones y composiciones para cada uno de estos momentos.

La musicoterapia tiene aplicaciones en diversos campos psicológicos. Por ejemplo, en la Unidad de Cuidados Intensivos (UCI) Pediátrica, \citeauthor{TRESIERRA:2005} (\citeyear{TRESIERRA:2005}) explica que la música actúa como un agente relajante, permitiendo al paciente comunicar sus emociones en un entorno donde se sienta escuchado. Además, el efecto relajante de la música está directamente relacionado con la percepción del dolor y el estrés, lo que puede tener un impacto positivo en el sistema inmunológico del paciente. Esta serie de reacciones en cadena mejora el bienestar biológico del paciente. Según \citeauthor{TRESIERRA:2005} (\citeyear{TRESIERRA:2005}), las terapias para niños autistas pueden beneficiarse de la musicoterapia. Esta puede romper el aislamiento social y mejorar el desarrollo socioemocional. En estos casos, un instrumento musical puede actuar como intermediario entre el paciente y el terapeuta, fortaleciendo su comunicación.

\subsubsection{Modalidades de musicoterapia}

La musicoterapia abarca diversos métodos y aplicaciones, agrupándose en dos modalidades distintas. La musicoterapia activa es aquella donde el paciente dirige la acción, combinándola normalmente con la creación musical como base fundamental. En contraste, la musicoterapia receptiva es aquella donde el paciente se convierte en un sujeto pasivo, como por ejemplo, escuchando cierto tipo de música relajante, y que impacta en sus emociones de manera inherente.

La musicoterapia activa conlleva que el paciente participe activamente en el proceso terapéutico, ya sea cantando, tocando un instrumento o bailando. En este escenario, el paciente se convierte en el intérprete o creador de la música, mientras que el terapeuta sirve como un acompañante del proceso. Un elemento crucial en la musicoterapia activa es la improvisación musical, la cual se define como el arte de crear música de manera espontánea al tocar. No se trata de crear música desde cero, sino más bien de responder a experiencias musicales previamente aprendidas. En estas terapias, el terapeuta no solo acompaña al paciente, sino que también puede participar activamente junto a él. Es fundamental que el terapeuta comprenda las intenciones creativas del paciente, ya que si no hay una conexión musical entre ambos, la musicoterapia activa no será efectiva (\cite{SALAMANCA:2003}).

En la musicoterapia receptiva o pasiva, dependiendo del contexto, a diferencia de la anterior modalidad, el paciente escucha música grabada o interpretada en vivo por el terapeuta. Aunque el paciente sigue participando en la sesión, su función cambia, convirtiéndose en un agente pasivo. Este tipo de musicoterapia puede aliviar el estrés y la ansiedad, mejorar la concentración y la creatividad, entre otros beneficios. Existen dos acercamientos comunes de la musicoterapia receptiva: el canto armónico y los cuencos tibetanos. El canto armónico es una hermosa forma de expresión que consiste en cantar dos, tres o incluso cuatro sonidos simultáneamente, utilizando la mayor cantidad posible de resonadores dentro del cuerpo y el cráneo. Por otro lado, los cuencos tibetanos están compuestos por hasta siete metales diferentes\footnote{Según las tradiciones, cada metal está relacionado con un astro diferente de nuestro sistema solar (\cite{SALAMANCA:2003}).}, que producen distintas resonancias armónicas. Estos cuencos, que se utilizan comúnmente para la meditación, se llenan con distintos niveles de agua para conseguir diferentes tipos de sonidos y efectos cuando se golpean o se frotan (\cite{SALAMANCA:2003}).

Como explica \citeauthor{BRUSCIA:1998-2} (\citeyear{BRUSCIA:1998-2}), en relación con ambas modalidades, cada sesión de musicoterapia involucra al paciente en algún tipo de experiencia musical. Estos diferentes tipos de experiencias musicales pueden incorporarse a las terapias de manera única o complementaria, según lo decida el terapeuta en función de las necesidades del paciente. Los cuatro tipos básicos son:

\begin{itemize}
	\item \textbf{Improvisación:} el paciente crea música ya sea cantando o tocando un instrumento de forma espontánea.
	\item \textbf{Recreación:} el paciente canta o toca una pieza musical preexistente, ya sea de memoria o con partitura.
	\item \textbf{Composición:} con la ayuda del terapeuta, el objetivo es componer y anotar una pieza musical en una partitura.
	\item \textbf{Escuchar:} el cliente escucha música, ya sea pregrabada o en vivo.
\end{itemize}

La improvisación, la recreación y la composición forman parte de la musicoterapia activa. Cada uno de estos enfoques ofrece una serie de beneficios únicos. Es útil implementar en la terapia cada uno de ellos de manera específica después de analizar el tipo de patología al que va dirigido. Sin embargo, la escucha es parte de la musicoterapia receptiva.

La música actúa como aliada en la terapia, colaborando con el terapeuta para intervenir conjuntamente. El terapeuta puede utilizar la música para alcanzar sus objetivos, ya sea por sí misma o en combinación con intervenciones personales. La naturaleza de las intervenciones musicales en la terapia puede variar según la estética que se desee transmitir al paciente. Sonido, belleza y creatividad se unen en la musicoterapia para ofrecer una metodología poderosa en la gestión emocional de los pacientes (\cite{BRUSCIA:1998-2}). Además, la musicoterapia es una opción poderosa en las terapias de grupo. Como explican \citeauthor{PRIETO:2022} (\citeyear{PRIETO:2022}), la música une a las personas sin importar la nacionalidad, el color o la raza. Contribuye a nuestra construcción y abre un camino de trascendencia más allá de nuestro cuerpo e intelecto. Independientemente de nuestra condición física o psicológica, la evocación emocional de la música es común a todas las personas.

\subsubsection{Beneficios de la musicoterapia en el tratamiento de la ansiedad}

La musicoterapia ha surgido como un tratamiento eficaz para gestionar la ansiedad, fundamentada por una creciente base de evidencia científica. Varios estudios y ensayos clínicos han demostrado que la musicoterapia puede disminuir significativamente los niveles de ansiedad en poblaciones de distintos contextos, desde pacientes hospitalizados hasta individuos con trastornos de ansiedad generalizados. Un estudio realizado por \citeauthor{SEPULVEDA:2014} (\citeyear{SEPULVEDA:2014}) en una población de niños de entre 8 y 16 años con diferentes tipos de cáncer, encontró una disminución significativa en los niveles de ansiedad tanto antes (prequimioterapia) como después (posquimioterapia) de la quimioterapia. No obstante, cuando se combinó esta terapia convencional con la escucha de música, la reducción en los niveles de ansiedad fue sustancialmente mayor. Para ser precisos, los niveles de ansiedad pasaron a reducirse del 27\% al 95\%.

\subsection{Videojuegos serios para la salud}

\subsubsection{Definción e historia de los videojuegos serios}

Los ''serious games'', o videojuegos serios, están diseñados con un propósito que va más allá del mero entretenimiento. Se utilizan en varios campos, como la educación, la salud, la formación militar y la simulación profesional, con el objetivo de alcanzar metas específicas en cada campo concreto. Estas pueden ser el desarrollo de habilidades, la alteración de comportamientos, o la terapia psicológica. A diferencia de los videojuegos tradicionales, que buscan principalmente entretener, los videojuegos serios tienen como objetivo educar, capacitar o generar cambios positivos en los usuarios.

Los videojuegos serios poseen una serie de características que los diferencian del formato convencional de los videojuegos, cuyo principal objetivo es entretener.

\begin{itemize}
	\item \textbf{Propósito educativo o terapéutico:} es evidente que el objetivo de los juegos serios es distintivo. A diferencia de los videojuegos convencionales, cuyo propósito es proporcionar entretenimiento, los juegos serios buscan impartir conocimientos o habilidades específicas a través de su diseño. Un estudio realizado por \citeauthor{GRAAFLAND:2012} (\citeyear{GRAAFLAND:2012}), analiza cómo los videojuegos serios pueden proporcionar un entorno seguro para que los profesionales practiquen cirugía en un entorno controlado sin poner en peligro a ningún paciente.
	\item \textbf{Interactividad:} al adaptarse al formato digital del videojuego convencional, la interacción con el medio ofrece un feedback inmediato. Como explican \citeauthor{CONNOLLY:2012} (\citeyear{CONNOLLY:2012}), esto permite que los usuarios aprendan rápidamente y sobre todo practicando en entornos cercanos a la realidad.
	\item \textbf{Enfoque en el usuario:} al estar completamente adaptados a las necesidades específicas del usuario, los videojuegos serios permiten una personalización que maximiza su efectividad para cumplir con su objetivo.
	\item \textbf{Evaluación y métricas:} el entorno digital permite realizar un seguimiento a tiempo real de las actividades del usuario. Además, la implementación de métodos para evaluar el progreso del usuario ofrecen al administrador la posibilidad de dirigir o ajustar las actividades según las necesidades que se observen. Este factor es crucial en entornos educativos y terapéuticos.
\end{itemize}

La historia de los videojuegos serios se remonta a la década de los años 60, con el desarrollo de simulaciones militares que buscaban entrenar a los soldados en tácticas y estrategias sin el riesgo de un combate real (\cite{SAWYER:2002}). Desde entonces, la evolución de estos juegos ha estado creciendo, expandiéndose a diversos campos y aprovechando los avances tecnológicos para crear experiencias cada vez más inmersivas y efectivas en su propio contexto. 

En la educación, los videojuegos serios comenzaron a ganar popularidad en la década de 1980 con juegos como ''The Oregon Trail'', que combinaba elementos de juego con lecciones de historia y supervivencia. En el ámbito de la salud, los años 2000 vieron un aumento en el desarrollo de juegos diseñados para la rehabilitación física y mental, como ''Re-Mission'', cuyo objetivo es ayudar a los pacientes jóvenes con cáncer a comprender y manejar su enfermedad. Hoy en día, los videojuegos serios utilizan tecnologías emergentes como la realidad virtual y aumentada para crear entornos de aprendizaje y terapia. Estas innovaciones permiten a los usuarios experimentar situaciones complejas y aprender de ellas en un entorno controlado y seguro (\cite{FREITAS:2011}).

\section{Trabajos relacionados}

Se decribirán en este apartado otros trabajos y desarrollos previos que hayan abordado cuestiones similares o relacionadas con los objetivos del trabajo actual.



% CAPÍTULO 3 - ASPECTOS METODOLÓGICOS
\chapter{Aspectos metodológicos}  
%\addcontentsline{toc}{chapter}{\numberline{}Aspectos metodológicos}
Justificación de las técnicas y tecnologías empleadas en el trabajo, especificando las razones por las que se han descartado otras igualmente aplicables.

\section{Metodología}

En este apartado se especificarán las fases del trabajo, así como la metodología o metodologías empleadas para desarrollar cada una de las fases. 

\newpage

\section{Tecnologías empleadas}

Las tecnologías utilizadas para el desarrollo de la aplicación se pueden clasificar según su uso. Debido a que el formato propuesto para el desarrollo es similar al de un videojuego, donde la interacción es esencial para su uso correcto y óptimo, se ha seleccionado software adecuado para el producto realizado. A continuación, se explicará detalladamente el uso de las diferentes tecnologías y las razones detrás de la elección de cada una específicamente.

\subsection{Unity}

Un motor de videojuegos, del inglés game engine, es un entorno que proporciona un conjunto de herramientas reutilizables que facilitan la creación de videojuegos a los desarrolladores. Estos se pueden dividir en el motor gráfico, responsable del aspecto visual, y el motor físico, encargado de dotar al motor de leyes físicas como la gravedad, la masa o las fuerzas. Cada motor de videojuegos tiene sus propios usos y limitaciones. Por lo tanto, la elección del motor a utilizar es de gran importancia antes de iniciar el desarrollo. Existen motores más especializados en el desarrollo 3D como \textit{Unreal Engine} (\cite{UE:1998}), mientras que otros están centrados en gráficos 2D, como \textit{Game Maker Studio} (\cite{GMS:1999}) y \textit{RPG Maker} (\cite{RPGM:1992}). \textit{Unity} (\cite{UNITY:2005}) y \textit{Godot Engine} (\cite{GODOT:2001}) son híbridos que permiten el desarrollo en ambas dimensiones.

Unity, junto con Unreal Engine y recientemente Godot Engine, es uno de los motores de videojuegos más populares en la industria. Aunque Unreal Engine es valorado como posiblemente el mejor motor de videojuegos de la actualidad, su uso está limitado a circunstancias muy específicas: videojuegos 3D para consolas de última generación. Sin embargo, Unity, aunque no ofrece la potencia gráfica de Unreal Engine, es mucho más versátil y se puede utilizar en una amplia variedad de circunstancias. Esto permite que el rango de plataformas para las que se puede desarrollar con este motor aumente, convirtiéndolo en una opción más segura para enfocarse en el desarrollo multiplataforma.

\begin{figure}[h!]
	\centering
	\subfigure[Unity.]{\includegraphics[width=0.2\textwidth]{./Figuras/Aspectos/UnityLogo}\label{fig:UnityLogo}}
	\hfil
	\subfigure[Unreal Engine.]{\includegraphics[width=0.2\textwidth]{./Figuras/Aspectos/UnrealLogo.png}\label{fig:UnrealLogo}}
	\hfil
	\subfigure[Godot Engine.]{\includegraphics[width=0.2\textwidth]{./Figuras/Aspectos/GodotLogo.png}\label{fig:GodotLogo}}
	\caption{Logotipos de los motores de videojuegos mencionados en el párrafo anterior.}
	\label{fig:GameEngineLogos}
\end{figure}

Hemos seleccionado Unity para nuestro caso específico por diversas razones, dado que la aplicación se desarrolla completamente en 2D y nuestra plataforma objetivo son los dispositivos móviles, en particular las tabletas. Unity domina el 50\% del desarrollo de videojuegos para móviles, consolas y PC, con un 71\% de los 1000 mejores juegos móviles creados en Unity (\cite{PLARIUM:2024}).

El motor gráfico 2D de Unity ofrece diversas herramientas útiles para el desarrollo, como el Sprite Editor. Además, Unity tiene una sólida integración de físicas 2D que permite simular colisiones, gravedad y otros elementos físicos. Su interfaz es intuitiva y fácil de usar, como se muestra en la \autoref{fig:UnityUI}, facilitando la creación rápida de videojuegos.

\begin{figure}[h!]
	\centering
	\includegraphics[width=0.8\linewidth]{./Figuras/Aspectos/UnityUI.png}
	\caption{Interfaz de usuario de la plantilla 2D de Unity.}
	\label{fig:UnityUI}
\end{figure}

La capacidad de arrastrar y soltar objetos en el editor de Unity simplifica significativamente el proceso de diseño, permitiendo a los desarrolladores centrarse en la creatividad y la jugabilidad en lugar de en aspectos técnicos complejos. Además, Unity cuenta con una comunidad amplia y activa que ofrece acceso a foros, tutoriales, cursos en línea y documentación extensa. Esta comunidad es un recurso invaluable para obtener soporte y encontrar soluciones a problemas comunes, aprender técnicas nuevas y compartir experiencias.

La Asset Store de Unity ofrece una gama variada de recursos como sprites, animaciones, scripts y plugins, que pueden agilizar el desarrollo. Estos recursos preconstruidos permiten a los desarrolladores integrar rápidamente elementos visuales y funcionales complejos sin la necesidad de crearlos desde cero, ahorrando tiempo y esfuerzo.

Todos estos factores, sumados a que Unity es el motor en el que más experiencia tenemos, fueron decisivos para seleccionarlo como el motor de desarrollo para ARTEMIS. La versión seleccionada es la 2021.3.31f, que fue la última versión de soporte a largo plazo (LTS) disponible en el momento en que se inició el proyecto.

\subsection{Visual Studio 2022}

Visual Studio es un entorno de desarrollo integrado (IDE) y editor de código, creado por Microsoft en 1997. Este editor es compatible con numerosos lenguajes de programación, como C++, Visual Basic .NET, Fortran, J\# y, más relevante para nuestro desarrollo, C\#. Su versión 2022 es la última versión disponible actualmente, y ha incorporado mejoras de rendimiento y nuevas funciones de accesibilidad para el usuario. Visual Studio, junto con Visual Studio Code y JetBrains Rider, forma el conjunto de editores de código con la mejor integración con Unity.

\subsection{FMOD}

FMOD Studio es un middleware\footnote{Un middleware, en términos de informática, es un software que facilita la comunicación entre las distintas aplicaciones y el sistema operativo. Además, proporciona funcionalidades inteligentes que permiten una innovación más rápida y un desarrollo eficiente. En el desarrollo de videojuegos, el middleware es un software que facilita la implementación de funcionalidades específicas dentro del motor.} de audio para videojuegos. Fue desarrollado y lanzado por Fireflight Technologies en 1995. Este motor de música y efectos de sonido se asemeja a un DAW tradicional, como se puede observar en la \autoref{fig:FMODUI}. FMOD Studio pertenece a la misma familia que el middleware Audiokinetic Wwise. Ambos proveen herramientas para la sonorización de videojuegos que facilitan la implementación de audio. Sin embargo, se ha decidido utilizar FMOD Studio debido a la mayor familiaridad con este software.

\begin{figure}[h!]
	\centering
	\includegraphics[width=0.6\linewidth]{./Figuras/Aspectos/FMODStudio.jpg}
	\caption{Interfaz de usuario de FMOD Studio.}
	\label{fig:FMODUI}
\end{figure}

A diferencia del sistema de audio de Unity, que tiene funcionalidades limitadas, FMOD Studio permite gestionar con gran precisión las pistas de audio en eventos, incluso utilizando filtros de diversos tipos si fuera necesario. Desde FMOD Studio, se pueden añadir parámetros asociados a zonas de bucle, compases musicales o pistas, que luego pueden ser modificados desde los propios componentes de Unity o por programación de scripts en función de las necesidades del videojuego que se esté desarrollando.

\subsection{TeXstudio}

TeXstudio es un editor de LaTeX de código abierto lanzado en 2009 que ofrece un soporte moderno para la escritura. Cuenta con funcionalidades como la corrección ortográfica interactiva, el plegado de código y el resaltado de sintaxis. La decisión de utilizar un editor de LaTeX en lugar de editores de texto convencionales, como Word, se basa en la magnitud del documento. La citación y referencia automáticas de figuras o tablas que ofrece LaTeX proporciona estabilidad y sostenibilidad al documento. Además, cuando se combina con la distribución Miktex, permite la instalación automática de paquetes y la actualización a las últimas versiones tanto de los paquetes como del propio TeXstudio.

\subsection{GitHub}

GitHub es un repositorio que utiliza el control de versiones de Git para alojar proyectos. Tiene integración directa con Unity y LaTeX, por lo que ha sido muy útil para permitir el trabajo remoto entre los distintos miembros del grupo de investigación e incluso para trabajar en distintos dispositivos sin necesidad de migrar manualmente el proyecto.

% CAPÍTULO 4 - DESARROLLO DEL TRABAJO
\chapter{Desarrollo del trabajo}  
%\addcontentsline{toc}{chapter}{\numberline{}Desarrollo del trabajo}
En este capítulo se describen los trabajos desarrollados por el propio alumno a partir del estado de la cuestión descrito previamente. La estructura de este capítulo depende mucho del tipo de trabajo realizado, debiendo adaptarse a éste. Es la parte más importante del TFG, y debería ser, por tanto, la desarrollada con mayor amplitud y detalle.


Algunos ejemplos de referencias son:

\begin{itemize}
\item Libro: \parencite{ASdV:2023}
\item Artículo en revista: \parencite{JLP:2011}
\item Artículo en actas de congreso: \parencite{OCC:2023}
\item Capítulo de libro: \parencite{Micciancio:2009:LBC}
\item Página web: \parencite{U-tad:2023}
\item Estándar: \parencite{NIST:2020}
\end{itemize}




% CAPÍTULO 5 - CONCLUSIONES
\chapter{Conclusiones}  
%\addcontentsline{toc}{chapter}{\numberline{}Conclusiones}
\section{Conclusión}

La ansiedad es una emoción común en la vida humana, presente desde la niñez hasta la vejez. Es probable que tengamos que lidiar con sus síntomas en algún momento de nuestras vidas. El proyecto ARTEMIS se creó para proporcionar soporte digital a las terapias psicológicas enfocadas en el tratamiento de las emociones con connotaciones negativas mediante la musicoterapia. Para esta etapa del proyecto en concreto, la investigación se centró alrededor del tratamiento de la ansiedad. Al inicio de este TFG, se fijaron unos objetivos generales y específicos que se alinearon con los del proyecto ARTEMIS. El objetivo principal se ha centrado en desarrollar una aplicación con enfoque de videojuego serio, basándose en los principios del proyecto. Los objetivos secundarios, en cambio, han incluido el estudio de estructuras musicales para identificar patrones que los pacientes puedan integrar en sus creaciones musicales, así como el análisis del impacto de las experiencias interactivas digitales en las prácticas tradicionales de musicoterapia.

A lo largo del trabajo se han cumplido los distintos objetivos en sus respectivas secciones. Tanto el estudio de las estructuras musicales como el análisis del impacto de las experiencias interactivas digitales en musicoterapia se han realizado en el marco teórico. Con el conocimiento de las figuras rítmicas y el proceso de composición, se pueden crear piezas musicales que relajen al paciente. Además, si la pieza es creada por el propio paciente, la satisfacción en su creación es subjetiva, facilitando el proceso de mejora. Las experiencias interactivas digitales no solo aumentan la interacción entre el terapeuta y el paciente, sino que también sitúan a ambos en una situación de doble usuario donde pueden cooperar para alcanzar el objetivo principal de la terapia. Es importante destacar que la adaptabilidad de este formato permite al terapeuta ajustar las terapias fácilmente según las necesidades del paciente. Lograr cumplir ambos objetivos desemboca en el desarrollo de la aplicación que no solo debe ser funcional desde un punto de vista terapéutico, sino que también debe tener la capacidad de establecer una conexión emocional entre el terapeuta y el paciente para facilitar una comunicación efectiva.

En rasgos generales, la investigación se enfocó en los pilares fundamentales del proyecto ARTEMIS: las emociones (específicamente la ansiedad), el ritmo y la creación musical, la musicoterapia y el desarrollo de videojuegos serios. Todos estos temas, que abarcan las áreas de conocimiento de la psicología, la música y los videojuegos, se exploraron a fondo en el marco teórico. En primer lugar, se definieron el ritmo y sus componentes principales (pulso, tempo y métrica) utilizando diversas fuentes. Citamos a \citeauthor{CASTELLANOS:2009} (\citeyear{CASTELLANOS:2009}) y su explicación de cómo el ritmo juega un papel esencial en la música, permitiendo marcar el tiempo de una pieza musical. Es similar a cómo miramos el reloj para realizar ciertas acciones. Además, el ritmo, siendo un movimiento cíclico, tiene una gran relación con la inhalación y exhalación de la respiración. Este aspecto confiere al ritmo un potencial terapéutico de grandes dimensiones.

Se exploró también el proceso de composición musical y las diversas maneras de componer música. Utilizamos el trabajo de \citeauthor{SUBIRATS:2004} (\citeyear{SUBIRATS:2004}) para investigar los diferentes niveles de creatividad que otorgan diversas habilidades en la creación. Es importante destacar la diferencia entre la improvisación libre y la composición guiada. Ambos tipos de creación son útiles en entornos terapéuticos y, combinados con las diferentes formas de generar música, brindan a los terapeutas un abanico de opciones para sus sesiones. Sin embargo, es necesario evaluar las necesidades del paciente para determinar qué tipo de creación musical se debe utilizar. En relación con el oyente de música, existe un término conocido como expectativa musical que juega un papel crucial en el manejo de las emociones. \citeauthor{SENABRE:2019} (\citeyear{SENABRE:2019}) expone una serie de teorías que fundamentan la expectativa musical. Estas teorías nos ayudan a comprender qué espera un paciente de su creación musical y cómo podemos ayudarle a alcanzar un resultado satisfactorio.

Para comprender la situación de un paciente con síntomas de ansiedad, fue necesario definir esta emoción, los tipos de trastornos que existen y sus tratamientos. Aunque nuestro caso de estudio está enfocado en las terapias psicológicas para tratar la ansiedad, también ampliamos la información para incluir tratamientos farmacológicos con el mismo objetivo. Se examinó la relación entre esta emoción y la musicoterapia. Antes de hacerlo, definimos qué es la musicoterapia. Nos basamos en \citeauthor{WFMT:2024} (\citeyear{WFMT:2024}), una federación especializada en musicoterapia, para comprender el verdadero significado de las terapias basadas en música. Observamos que la musicoterapia se divide en dos modalidades: activa y receptiva. En la musicoterapia activa, el paciente se encuentra en un contexto donde él es el creador de la música, mientras que en la musicoterapia receptiva, el paciente escucha música grabada o interpretada en vivo por el terapeuta. La elección entre una u otra modalidad recae en el terapeuta, quien debe evaluar las circunstancias del paciente para tomar tal decisión. Se ha demostrado que la música es un componente poderoso en el tratamiento de la ansiedad. Un estudio realizado por \citeauthor{SEPULVEDA:2014} (\citeyear{SEPULVEDA:2014}) mostró que la musicoterapia reducía drásticamente los niveles de ansiedad en pacientes infantiles con cáncer.

Finalmente, se exploraron los elementos de diseño que definen al videojuego serio, enfatizando las características que lo distinguen del formato tradicional del videojuego. Se especificó en los diferentes campos de aplicación de los videojuegos serios, como la educación o la salud, proporcionando ejemplos de desarrollos específicos cuyo objetivo era otorgar una mejora a los usuarios en cada campo particular. La investigación culminó con una recopilación de videojuegos serios para la salud, detallando sus objetivos, los elementos de diseño que incluyen y cómo podrían ser útiles en nuestro desarrollo. \textit{Operation Quest} (\cite{OPERATIONQUEST:2024}), por su enfoque hacia el público infantil; \textit{SPARX} (\cite{SPARX:2013}), por su narrativa que explica al paciente su situación; \textit{Flow} (\cite{FLOW:2006}), por su simplicidad mecánica; y \textit{Deep VR} (\cite{DEEP:2021}), por su ambiente inmersivo e innovadoras aplicaciones de interacción, proporcionan una perspectiva amplia de los objetivos del proyecto ARTEMIS.

Se decidió adoptar una metodología que imita los procesos de desarrollo de un videojuego tradicional. Las características principales en la creación de un videojuego incluyen la definición de requisitos antes de comenzar, la conceptualización y diseño de las mecánicas del juego, y el desarrollo iterativo. En nuestro caso específico, para ofrecer un diseño de juego sólido, es necesario fundamentar todas las decisiones de diseño en las investigaciones realizadas previamente. Hemos utilizado tecnologías que facilitan la creación de experiencias digitales interactivas, como Unity, la gestión de audio a través de FMOD, y el trabajo colaborativo en repositorios como GitHub. En combinación con la metodología estipulada, estas herramientas nos han permitido desarrollar una aplicación interactiva de musicoterapia, sin requerir que el equipo esté en el mismo lugar o tiempo. Aplicando esta metodología, se logró desarrollar un sistema interactivo en el que el paciente puede disfrutar de la creación musical como si resolviera un rompecabezas sonoro. La solución se encuentra en la satisfacción personal del paciente.

En conclusión, analizar el impacto de las experiencias digitales interactivas en las sesiones de musicoterapia mostró que estas no solo proporcionan un medio adaptativo que permite al terapeuta personalizar las terapias según las necesidades del paciente, sino que también sumergen al paciente en un entorno visual y sonoro artístico que puede ser más efectivo que las terapias tradicionales. Este formato permite al terapeuta tener un registro en tiempo real de la interacción del paciente con la aplicación, lo que posibilita el análisis de las necesidades del paciente al instante. Para el usuario, los efectos de la terapia son inmediatos debido a la interacción instantánea. En cuanto al desarrollo de la aplicación, que se basa en la modalidad de musicoterapia activa, se fundamenta en la investigación realizada. Combina las contribuciones de diferentes referencias para crear un puzle sonoro relajante en el que los pacientes pueden disfrutar creando música y asienta las bases de una posible ampliación futura hacia un alcance de emociones más allá de la ansiedad.

Este Trabajo Final de Grado no tiene pruebas empíricas que demuestren la utilización de esta aplicación en situaciones con pacientes reales; se fundamenta únicamente en las investigaciones realizadas. Aunque la intención inicial era utilizar la aplicación en terapias reales llevadas a cabo por los psicólogos asociados al proyecto ARTEMIS, la falta de tiempo impidió que esto ocurriera en esta primera etapa del proyecto. Haber podido comprobar la eficacia y recibir retroalimentación de los pacientes infantiles que sufren síntomas de algún tipo de ansiedad, nos habría permitido crear un producto final más iterado y pulido.

\section{Líneas de investigación futuras}

Aunque el proyecto ARTEMIS se ha centrado inicialmente en la ansiedad, tiene como objetivo expandir gradualmente su enfoque para tratar otras emociones a través de la musicoterapia. El desarrollo de esta aplicación establece las bases para un sistema complementario a las sesiones de terapia tradicional del terapeuta. El uso de un medio digital interactivo para el desarrollo ofrece un comportamiento modular que permite una gran escalabilidad del proyecto. Con el arte y las líneas narrativas ya definidos, la investigación de emociones adicionales solo requerirá identificar los problemas asociados con esas emociones y cómo tratarlos. El contexto artístico y narrativo ya está establecido.

%\clearpage
%\thispagestyle{empty}
%\printindex \nocite{*}
%\appendix

% REFERENCIAS
\newpage{\pagestyle{empty}}
\addcontentsline{toc}{chapter}{\numberline{}Referencias}
%https://github.com/SantiPlanet/apalike
%\bibliographystyle{apalike-es} 
%\bibliography{biblio}
\printbibliography

% ANEXO
\newpage{\pagestyle{empty}}
\markboth{Anexo}{Anexo}
\chapter*{Anexo}  
\addcontentsline{toc}{chapter}{\numberline{}Anexo}    
\section{Sistema de cuadrícula}

\subsection{Objeto cuadrícula} \label{code:grid}

\begin{lstlisting}
using System.Collections;
using System.Collections.Generic;
using UnityEngine;

public class Grid
{
	private int width;
	private int height;
	private float cellSize;
	private int[,] gridArray;
	
	public Grid(int width, int height, float cellSize)
	{
		this.width = width;
		this.height = height;
		this.cellSize = cellSize;
		
		gridArray = new int[width, height];
		
		for (int x = 0; x < gridArray.GetLength(0); x++)
		{
			for (int y = 0; y < gridArray.GetLength(1); y++)
			{
				Utils.CreateWorldText(gridArray[x, y].ToString(), null, GetWorldPosition(x, y), 20, Color.white, TextAnchor.MiddleCenter, TextAlignment.Center, 0);
			}
		}    
	}
	
	public Vector3 GetWorldPosition(int x, int y)
	{
		return new Vector3(x, y) * cellSize;
	}
}
\end{lstlisting}

\subsection{Clase ''Utils''} \label{code:gridUtils}

\begin{lstlisting}
using System.Collections;
using System.Collections.Generic;
using UnityEngine;

public static class Utils
{
	public static TextMesh CreateWorldText(string text, Transform parent, Vector3 localPosition, int fontSize, Color color, TextAnchor textAnchor, TextAlignment textAlignment, int sortingOrder)
	{
		if (color == null)
		{
			color = Color.white;
		}
		
		return CreateWorldText(parent, text, localPosition, fontSize, color, textAnchor, textAlignment, sortingOrder);
	}
	
	public static TextMesh CreateWorldText(Transform parent, string text, Vector3 localPosition, int fontSize, Color color, TextAnchor textAnchor, TextAlignment textAlignment, int sortingOrder)
	{
		GameObject gameObject = new GameObject("WorldText", typeof(TextMesh));
		Transform transform = gameObject.transform;
		transform.SetParent(parent, false);
		transform.localPosition = localPosition;
		
		TextMesh textMesh = gameObject.GetComponent<TextMesh>();
		textMesh.anchor = textAnchor;
		textMesh.alignment = textAlignment;
		textMesh.text = text;
		textMesh.fontSize = fontSize;
		textMesh.color = color;
		textMesh.GetComponent<MeshRenderer>().sortingOrder = sortingOrder;
		
		return textMesh;
	} 
}
\end{lstlisting}


\begin{lstlisting}
	using System.Collections;
	using System.Collections.Generic;
	using UnityEngine;
	
	public class GameManager : TemporalSingleton<GameManager>
	{
		[Header("Cannon")]
		[SerializeField] private Transform cannonTransform;
		[SerializeField] private float cannonForce;
		
		[SerializeField] GameObject temporalCharacter;
		
		public void Start()
		{
			ResetTransform(temporalCharacter);
		}
		
		public void ResetTransform(GameObject gameObject) {
			gameObject.transform.position = cannonTransform.position;
			gameObject.transform.rotation = cannonTransform.rotation;
			
			if (gameObject.GetComponent<Rigidbody2D>() != null) { 
				ResetGravity(gameObject.GetComponent<Rigidbody2D>());
				Launch(gameObject.GetComponent<Rigidbody2D>()); 
			}
		}
		
		private void ResetGravity(Rigidbody2D cmpRb) { cmpRb.velocity = Vector3.zero; }
		
		private void Launch(Rigidbody2D cmpRb) { cmpRb.AddForce(cannonTransform.up * cannonForce, ForceMode2D.Impulse); }
	}
\end{lstlisting}

\end{document}
