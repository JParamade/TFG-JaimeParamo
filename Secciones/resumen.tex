El objetivo del proyecto ARTEMIS es desarrollar una aplicación que sirva como herramienta complementaria en las sesiones de musicoterapia que tienen como finalidad ayudar a los pacientes a pasar de un estado de ansiedad a uno de calma. Para lograr este objetivo, se ha investigado sobre el funcionamiento de la ansiedad, sus posibles tratamientos, las terapias de musicoterapia y sus aplicaciones en el campo psicológico relevante, así como los elementos de diseño de videojuegos serios orientados a la salud. Se ha desarrollado una aplicación que abarca diversas actividades con objetivos comunes, utilizando tecnologías que permiten la creación de experiencias digitales interactivas como Unity, la gestión de audio como FMOD, y el trabajo colaborativo en repositorios como GitHub. En concreto, la actividad que se ha desarrollado en este trabajo, sumerge al paciente en una relajante experiencia rítmica en el que tendrá que encontrar una solución subjetiva a un puzle musical. La elaboración de esta experiencia siguiendo las bases terapéuticas de relajación y la interactividad del medio digital centrado en el usuario, permite al terapeuta adaptarse a las necesidades de cada paciente, e incluso a las diferentes necesidades de un mismo paciente en distintas etapas de la terapia.

\textit{Palabras clave:} videojuegos, musicoterapia, ritmo, salud mental, ansiedad, ARTEMIS