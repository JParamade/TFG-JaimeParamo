Esta investigación está vinculada al proyecto ARTEMIS, un proyecto de investigación enfocado en el desarrollo de experiencias digitales interactivas basadas en musicoterapia. Estas experiencias se agrupan en una aplicación diseñada para funcionar como un puente entre el médico y el paciente en terapias psicológicas. El objetivo es ayudar a los pacientes a transitar de emociones con connotaciones negativas hacia emociones con connotaciones positivas.

La investigación se ha centrado en la ansiedad infantil y su tratamiento mediante la musicoterapia. Sin embargo, el proyecto ARTEMIS aspira a ser escalable para diferentes grupos de edad y emociones. Inicialmente, se consideró enfocar la investigación en grupos vulnerables como niños y ancianos. No obstante, se encontró que los niños interactúan más fácilmente con entornos digitales, por lo que se decidió que ese sería el punto de partida para ARTEMIS.

Se ha desarrollado e integrado una experiencia en la aplicación con el objetivo de facilitar la transición de un estado de ansiedad a la calma. A lo largo del documento, se explicará cómo se logra esta transición a través de una ordenación rítmica que pasa del caos al orden subjetivo, mediante la interacción del paciente con las instrucciones proporcionadas por el terapeuta.

\begin{figure} [h!]
	\centering
	\includegraphics[width=0.9\linewidth]{Figuras/Introduccion/1_LogoArtemis}
	\caption{Logotipo del proyecto ARTEMIS.}
	\label{fig:logoArtemis}
\end{figure}

\section{Justificación y contexto}

El proyecto ARTEMIS se crea con el propósito de proporcionar soporte digital a las terapias psicológicas que tienen como objetivo utilizar la musicoterapia como medio para facilitar la transición entre estados emocionales. Su función radica en ser una herramienta complementaria que los terapeutas pueden implementar junto a las terapias más tradicionales, adaptándola a las necesidades de cada paciente. La aplicación, que conglomera las distintas experiencias, no debe ser utilizada solo por el paciente. Tiene un enfoque de doble usuario, donde ambas partes tienen su rol definido en la terapia. El paciente interactúa con la aplicación siguiendo las indicaciones del terapeuta, quién recogerá información relevante tanto antes como después de la interacción. De este modo, puede recibir feedback a tiempo real y redirigir las sesiones terapéuticas si fuera necesario. Además, el terapeuta tiene acceso a un registro que le permite observar la progresión del paciente. Esto le ayuda a identificar patrones y a priorizar los elementos que funcionan mejor en cada caso específico.

La investigación busca aportar una síntesis entre el diseño de productos interactivos y la musicoterapia, fundamentada por estudios previos sobre el uso de la música en el tratamiento de trastornos emocionales, en particular, la ansiedad. La elección de la ansiedad como emoción de estudio, es \parencite{CJI:2017}.


\section{Motivación}

Aquí va la motivación.

\section{Planteamiento del problema}

En este apartado se describe, con toda claridad, la problemática que se va a investigar atendiendo especialmente a las causas y consecuencias del objeto de estudio escogido para el TFG. A veces, el planteamiento del problema exige el desarrollo de sub-apartados.

\section{Objetivos del trabajo}

Una vez planteado el problema, se programarán los objetivos que se pretenden alcanzar en el trabajo.