Esta investigación está vinculada al proyecto ARTEMIS, un proyecto de investigación enfocado en el desarrollo de experiencias digitales interactivas basadas en musicoterapia. Estas experiencias se agrupan en una aplicación diseñada para funcionar como un puente entre el médico y el paciente en terapias psicológicas. El objetivo es ayudar a los pacientes a transitar de emociones con connotaciones negativas hacia emociones con connotaciones positivas.

La población de estudio en la que se ha centrado la investigación es  

\begin{figure} [h!]
	\centering
	\includegraphics[width=0.7\linewidth]{Figuras/Introduccion/1_LogoArtemis}
	\caption{Logotipo proyecto ARTEMIS.}
	\label{fig:logoArtemis}
\end{figure}

\section{Justificación y contexto}

El proyecto ARTEMIS nace con la intención de dar soporte digital a las terapias

Musicoterapia digital.

\section{Motivación}

Aquí va la motivación.

\section{Planteamiento del problema}

En este apartado se describe, con toda claridad, la problemática que se va a investigar atendiendo especialmente a las causas y consecuencias del objeto de estudio escogido para el TFG. A veces, el planteamiento del problema exige el desarrollo de sub-apartados.

\section{Objetivos del trabajo}

Una vez planteado el problema, se programarán los objetivos que se pretenden alcanzar en el trabajo.