Esta investigación está vinculada al proyecto ARTEMIS, un proyecto de investigación enfocado en el desarrollo de experiencias digitales interactivas basadas en musicoterapia. Estas experiencias se agrupan en una aplicación diseñada para funcionar como un puente entre el médico y el paciente en terapias psicológicas. El objetivo es ayudar a los pacientes a transitar de emociones con connotaciones negativas hacia emociones con connotaciones positivas.

La investigación se ha centrado en la ansiedad infantil y su tratamiento mediante la musicoterapia. Sin embargo, el proyecto ARTEMIS aspira a ser escalable para diferentes grupos de edad y emociones. Inicialmente, se consideró enfocar la investigación en grupos vulnerables como niños y ancianos. No obstante, se encontró que los niños interactúan más fácilmente con entornos digitales, por lo que se decidió que ese sería el punto de partida para ARTEMIS.

Se ha desarrollado e integrado una experiencia en la aplicación con el objetivo de facilitar la transición de un estado de ansiedad a la calma. A lo largo del documento, se explicará cómo se logra esta transición a través de una ordenación rítmica que pasa del caos al orden subjetivo, mediante la interacción del paciente con las instrucciones proporcionadas por el terapeuta.

\begin{figure} [h!]
	\centering
	\includegraphics[width=0.9\linewidth]{Figuras/Introduccion/1_LogoArtemis}
	\caption{Logotipo del proyecto ARTEMIS.}
	\label{fig:logoArtemis}
\end{figure}

\section{Justificación y contexto}

El proyecto ARTEMIS se crea con el propósito de proporcionar soporte digital a las terapias psicológicas que tienen como objetivo utilizar la musicoterapia como medio para facilitar la transición entre estados emocionales. Su función radica en ser una herramienta complementaria que los terapeutas pueden implementar junto a las terapias más tradicionales, adaptándola a las necesidades de cada paciente. La aplicación, que conglomera las distintas experiencias, no debe ser utilizada solo por el paciente. Tiene un enfoque de doble usuario, donde ambas partes tienen su rol definido en la terapia. El paciente interactúa con la aplicación siguiendo las indicaciones del terapeuta, quién recogerá información relevante tanto antes como después de la interacción. De este modo, puede recibir feedback a tiempo real y redirigir las sesiones terapéuticas si fuera necesario. Además, el terapeuta tiene acceso a un registro que le permite observar la progresión del paciente. Esto le ayuda a identificar patrones y a priorizar los elementos que funcionan mejor en cada caso específico.

La investigación busca aportar una síntesis entre el diseño de productos interactivos y la musicoterapia, fundamentada por estudios previos sobre el uso de la música en el tratamiento de trastornos emocionales, en particular, la ansiedad. Esta emoción fue seleccionada para su estudio debido a su prevalencia en la sociedad y a su presencia a lo largo de la vida de una persona, teniendo un gran impacto, especialmente en la etapa de la niñez y adolescencia. Un estudio conducido por \citeauthor{MO:2012} (\citeyear{MO:2012}) indica que, de una muestra de aproximadamente 2500 niños y adolescentes entre 8 y 17 años (51\% de sexo femenino), de diversas nacionalidades (aunque predominantemente española) y diferentes situaciones socioeconómicas, el 26,41\% presentó puntuaciones altas en algún tipo de ansiedad. Aunque se hable de esta emoción como una única, \citeauthor{CAGD:2010} (\citeyear{CAGD:2010}) explica que se caracteriza por ser un conjunto de diferentes tipos que surgen como reacción a eventos o situaciones. Estos tipos están condicionados por las circunstancias que rodean al individuo y los recursos que tiene, conocidos como estrategias de afrontamiento. Entre las reacciones a la ansiedad se incluyen la incertidumbre, la impotencia y la activación fisiológica, fuertemente relacionadas con síntomas corporales de tensión y preocupación acerca del futuro. Los síntomas pueden intensificarse significativamente cuando se combina con la depresión, llegando incluso a la somatización. Como explica \citeauthor{CJI:2017} (\citeyear{CJI:2017}), la ansiedad puede hacer que la mente enferme al cuerpo. Aunque aún no podemos explicar el origen de las causas, el dolor y sufrimiento se manifiestan de forma real. 

La música y la salud están estrechamente relacionadas. La combinación adecuada de elementos musicales como tonos, ritmos y armonías puede impactar positivamente las condiciones físicas, fisiológicas y psicológicas de las personas. Aquí es donde entra la musicoterapia, un enlace entre el médico y el paciente que busca cumplir un objetivo terapéutico específico a través de la interacción con medios musicales. En nuestro caso de estudio, el objetivo es reducir la ansiedad. \citeauthor{KTN:2011} (\citeyear{KTN:2011}) explica que los patrones de las ondas cerebrales cambian según el estado de ánimo del paciente que, combinados con ciertos tipos de música, pueden generar un equilibrio que conduce a la relajación. En los últimos años, se han empezado a implementar nuevas estrategias dentro de las sesiones de musicoterapia, incluyendo el uso de videojuegos como medio interactivo.

\section{Motivación}

La elección de la temática para este Trabajo de Fin de Grado se debe a la fusión de dos de mis mayores pasiones: los videojuegos y la música. Desde una edad temprana, ambos han jugado un papel crucial en mi vida, no solo como formas de entretenimiento, sino como medios de expresión y canales que me han servido para desarrollar la creatividad. La influencia personal de estas dos áreas se puede apreciar tanto desde la perspectiva del desarrollador o intérprete, como desde el punto de vista del jugador u oyente. El diseño y desarrollo de experiencias interactivas para la musicoterapia es el punto de convergencia de estas dos pasiones, donde se agrega el elemento psicológico, un tema de especial interés para mí. 

A través de la investigación y desarrollo de esta experiencia interactiva, pretendo potenciar mis conocimientos en los campos específicos a los que me gustaría dedicarme profesionalmente en la industria del videojuego. Este proyecto no solo me permitirá aplicar las habilidades técnicas y creativas que he ido desarrollando a lo largo de mi carrera educativa, sino que también me proporcionará una comprensión más profunda de cómo la música puede ser utilizada de manera terapéutica dentro de estos entornos interactivos. Además, me ayudará a comprender cómo, recopilar y analizar datos sobre la respuesta del usuario y los resultados terapéuticos, permite adaptar las sesiones terapéuticas al paciente a tiempo real, lo que fortalecerá mis habilidades de evaluación.

Mi gran curiosidad por la mente humana, que considero una pieza clave en la máquina perfecta que es el cuerpo humano, me impulsa a explorar cómo estas aplicaciones interactivas pueden influir en el cerebro y el comportamiento. Este proceso también me aportará información que alimente esta curiosidad en los campos psicológicos. Al investigar cómo las experiencias interactivas basadas en la música pueden influir psíquica y fisiológicamente, espero descubrir nuevas maneras en las que la tecnología puede ser utilizada para mejorar la salud mental y el bienestar general. Esta exploración que entrelaza disciplinas no solo enriquecerá mi formación académica y profesional, sino que también contribuirá a un campo de estudio en constante evolución, con el potencial de tener un impacto positivo y duradero en la vida de las personas.

\section{Planteamiento del problema}

En este apartado se describe, con toda claridad, la problemática que se va a investigar atendiendo especialmente a las causas y consecuencias del objeto de estudio escogido para el TFG. A veces, el planteamiento del problema exige el desarrollo de sub-apartados.

\section{Objetivos del trabajo}

Una vez planteado el problema, se programarán los objetivos que se pretenden alcanzar en el trabajo.
