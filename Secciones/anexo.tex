A continuación se muestran algunos elementos básicos en Latex.

\section*{Figuras}

\vspace{0.3cm}
\begin{figure}[!h]
	\centering
	\includegraphics[scale=0.5]{./Figuras/LogoUtad.png}
	\caption{Ejemplo de figura con imagen en formato PNG.}
	\label{fig:01}
\end{figure}


\section*{Tablas}

\begin{table}[!htbp]
	\begin{center}
		\begin{tabular}{| l | c | r |}
			\hline
			Clase & Aula & Cuatrimestre \\ 
			\hline
			\hline
			Cálculo & MD203 & Q1 \\
			\hline
			Análisis Matemático I & BE011 & Q2 \\
			\hline
			Álgebra & BE012 & Q1 \\ 
			\hline
		\end{tabular}
		\caption{Ejemplo de tabla}
		\label{tab:fruta}
	\end{center}
\end{table} 

\section*{Listas}

Lista numerada:
\begin{enumerate}
\item Primer elemento.
\item Segundo elemento.
\end{enumerate}

Lista sin numerar:
\begin{itemize}%[leftmargin=2mm,labelindent=12.5mm,labelsep=6.3mm]
	\item Primer elemento.
	\item Segundo elemento.
\end{itemize}

\section*{Fórmulas matemáticas}

Se pueden utilizar expresiones como $\sen(x)$, $y=\log(x)$, $\arctan(x)$ dentro de las frases, o crear fórmulas en párrafo separado:

\[
\int_{a}^{b} f(x) \, dx
\]

Es importante darse cuenta de la diferencia entre $\sum_{i=1}^{n} (2+i)^3$ y $\displaystyle  \sum_{i=1}^{n} (2+i)^3$.
