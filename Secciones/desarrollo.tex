El proyecto de investigación ARTEMIS es una iniciativa interdisciplinaria que reúne el trabajo de varios estudiantes de diferentes especialidades y campos. Este enfoque colaborativo permite abordar el proyecto desde diversas perspectivas, enriqueciendo así el desarrollo y la implementación de soluciones innovadoras. ARTEMIS se centra en el diseño y desarrollo de aplicaciones digitales interactivas. Estas han sido específicamente desarrolladas para realizar terapias de estimulación emocional a través del arte y la música, utilizando la tecnología como vehículo principal. Además, estas aplicaciones están diseñadas para ser intuitivas y accesibles, permitiendo a los usuarios interactuar con ellas de manera sencilla. Este proyecto tiene como objetivo integrar las últimas innovaciones tecnológicas con prácticas terapéuticas tradicionales, para potenciar los beneficios emocionales y psicológicos de los pacientes. A pesar de que cada uno de los estudiantes que conforman el equipo se ha centrado en líneas de investigación particulares, todos comparten objetivos comunes que han guiado sus esfuerzos hacia una meta colectiva.

Para contribuir al conocimiento científico y hacer funcionales las terapias diseñadas, las aplicaciones permiten la configuración por edades y patologías. Se definen pruebas psicométricas\footnote{Los test psicométricos son largos ya que consisten en alrededor de 30-40 preguntas. Al final de cada aplicación o sesión, se realizan preguntas rápidas del tipo termómetro.} al inicio y al final de la actividad, y los datos recogidos\footnote{La definición del perfil del paciente implica una entrevista entre el terapeuta y el paciente, donde se indaga acerca de su historia musical, gustos y características relacionadas con la música. El terapeuta será el encargado de introducir una serie de datos en la aplicación para optimizar esta fase.} de los pacientes se almacenan para su posterior consulta por el terapeuta. Los datos recogidos incluyen la edad, el nivel escolar, si se ha estudiado algún instrumento y el instrumento preferido del paciente. El perfil del paciente determina la complejidad musical en, por ejemplo, la composición interactiva. Un perfil con menos experiencia musical tendrá la capacidad de entender una complejidad musical menor o menos disonante, mientras que un perfil con más experiencia puede manejar una armonía más elaborada. Es importante mencionar que estas aplicaciones tienen una función de doble usuario en la que tanto el terapeuta como el paciente participan simultáneamente, cada uno con un rol específico. El terapeuta actúa como instructor, guiando al paciente en su interacción con la aplicación. 

La primera fase del proyecto ARTEMIS se ha enfocado en el tratamiento de la ansiedad infantil, utilizando la tecnología como vehículo para la transición emocional, complementándose con terapias tradicionales basadas en musicoterapia. La estructura de las sesiones terapéuticas es modular y se puede adaptar a las necesidades individuales del paciente. Cada sesión comienza con la definición del perfil del paciente y luego se selecciona una o más actividades vinculadas directamente a las líneas de investigación previamente definidas, en las que profundizaremos más en detalle en los próximos párrafos. Dentro de cada actividad, se han desarrollado una o más aplicaciones con distintos enfoques, que pueden ser utilizadas en función de las observaciones del terapeuta. Las actividades que han servido como punto de partida para el desarrollo de estas aplicaciones, y que han dado lugar a las líneas de investigación, son:

\begin{itemize}
	\item \textbf{Actividad 1 - Psicoeducación emocional:} esta primera actividad se enfoca en entender la emoción, específicamente la ansiedad, y en cómo combatirla. El educador, que en este caso es el terapeuta, proporciona de manera concisa información científica relevante para responder a preguntas importantes. El paciente recibe una serie de herramientas que le ayudan a identificar, comprender y manejar la ansiedad.
	\item \textbf{Actividad 2 - Relajación guiada mediante la respiración:} el terapeuta, con la ayuda del soporte digital, guía al paciente a realizar ejercicios de relajación centrados en la respiración. El paciente sincroniza sus respiraciones con la interacción tecnológica, aportando un elemento artístico que acompaña la respiración.
	\item \textbf{Actividad 3 - Ordenar pensamientos a través del ritmo:} el paciente se centra en utilizar el ritmo como herramienta para organizar sus pensamientos y emociones. Bajo la guía del terapeuta, se establece un escenario rítmico que permite al paciente explorar y organizar sus ideas y sentimientos de una manera ordenada, respetando siempre su subjetividad.
	\item \textbf{Actividad 4 - Improvisación:} se basa en la completa libertad del paciente, donde, a través de la improvisación musical, el paciente debe expresar su estado de ánimo.
	\item \textbf{Actividad 5 - Repetición de patrones rítmicos:} el paciente debe escuchar y repetir patrones rítmicos con la mayor precisión posible con el objetivo de controlar la impulsividad. 
\end{itemize}

Estas actividades constituyen la base sobre la que se han construido para las distintas aplicaciones, respaldadas por las líneas de investigación. Estas líneas se han establecido para abordar los objetivos del proyecto desde una perspectiva multidisciplinaria, uniendo las cuatro áreas de estudio: arte, tecnología, música y terapias de estimulación emocional. Las líneas de investigación son las siguientes:

\begin{itemize}
	\item Musicoterapia y narrativa audiovisual.
	\item Musicoterapia y arte digital.
	\item Composición interactiva y musicoterapia.
	\item Musicoterapia y diseño de videojuegos (serious games).
	\item UX/UI interacción y branding.
\end{itemize}

Examinaremos cada línea de investigación individualmente, indagando en profundidad en sus aplicaciones derivadas. Específicamente, la aplicación más relevante para este Trabajo de Fin de Grado se encuentra en la línea de investigación de diseño de videojuegos con musicoterapia. Dejaremos esta para el final con el fin de explicarla en mayor detalle, ya que es él núcleo principal de este trabajo.

\section{Musicoterapia y narrativa audiovisual}

La narrativa audiovisual es una parte integral del proyecto de investigación. Está presente en todas las aplicaciones, en mayor o menor medida, y actúa como un vínculo entre todas ellas. El objetivo de contar una historia es proporcionar información al paciente de manera que pueda sentirse identificado y conectar directamente con sus emociones. Dado que estamos estudiando la ansiedad, es importante explicar al paciente cómo funciona e involucrarlo en la explicación integrada en forma de historia.