Este apartado es esencial para relacionar la temática del TFG con la comunidad científica a la que pertenece, con las fuentes bibliográficas y la literatura del tema. El estado de la cuestión debe describir el estado actual de las investigaciones y/o desarrollos en torno al tema (se recomienda citar autores que hayan trabajado en el mismo campo).

\section{Marco teórico del trabajo}

\subsection{Musicoterapia y ansiedad}

\subsubsection{¿Qué es la ansiedad?}

La ansiedad es una emoción humana común que todos experimentamos en ciertos momentos de nuestra vida. Es una respuesta natural del cuerpo a situaciones de peligro o estrés y puede manifestarse de diversas maneras. Para comprender mejor la ansiedad, es importante abordar tanto su definición como los tipos específicos de trastornos de ansiedad.

La ansiedad se define generalmente como una sensación de temor, preocupación o malestar respecto a eventos futuros inciertos. A nivel fisiológico, la ansiedad activa el sistema nervioso simpático, lo que provoca una serie de respuestas corporales, como el aumento del ritmo cardíaco, la sudoración y la tensión muscular (American Psychiatric Association, 2013). Desde un punto de vista emocional, la ansiedad puede causar una sensación de aprehensión o temor, y cognitivamene, puede llevar a pensamientos intrusivos y rumiantes sobre posibles peligros o fracasos.

Según la American Psychological Association (APA, 2020), la ansiedad se diferencia del miedo en que esta es una respuesta a una amenaza inmediata, mientras que la ansiedad es la anticipación de una amenaza futura. Es un estado emocional que prepara al individuo para enfrentar potenciales desafíos o peligros. Esta respuesta puede ser adaptativa cuando se presenta en niveles moderados, ya que puede mejorar el rendimiento y la atención. Sin embargo, cuando la ansiedad es excesiva y persistente, puede interferir significativamente en la vida diaria de una persona.

Los síntomas de ansiedad pueden ser categorizados en tres dimensiones: físicos, emocionales y cognitivos.
\begin{itemize}
	\item \textbf{Síntomas Físicos:} Estos incluyen palpitaciones, sudoración, temblores, tensión muscular, dificultad para respirar, mareos, fatiga y problemas gastrointestinales. Estos síntomas son el resultado de la activación del sistema nervioso simpático, que prepara al cuerpo para la "lucha o huida" en situaciones percibidas como peligrosas (Craske \& Stein, 2016).
	\item \textbf{Síntomas Emocionales:} Las personas con ansiedad suelen experimentar una sensación persistente de temor o aprensión. Pueden sentirse inquietos, irritables, y tener una sensación de pérdida de control sobre sus emociones (Barlow, 2002).
	\item \textbf{Síntomas Cognitivos:} La ansiedad puede afectar la manera en que las personas piensan, llevándolas a tener pensamientos intrusivos, preocupaciones excesivas y dificultades para concentrarse. Estos síntomas cognitivos pueden perpetuar el ciclo de ansiedad, ya que la preocupación constante sobre posibles peligros puede intensificar los síntomas físicos y emocionales (Clark \& Beck, 2010).
\end{itemize}

\subsubsection{Tipos de Trastornos de Ansiedad}
Existen varios tipos de trastornos de ansiedad, cada uno con sus propias características y criterios diagnósticos. Los trastornos de ansiedad más comunes incluyen:

Trastorno de Ansiedad Generalizada (TAG): Se caracteriza por una preocupación excesiva e incontrolable sobre diversas actividades o eventos, acompañada de síntomas físicos como tensión muscular y problemas de sueño. Las personas con TAG suelen anticipar desastres y están constantemente preocupadas por su salud, dinero, familia o trabajo (American Psychiatric Association, 2013).

Trastorno de Pánico: Este trastorno se caracteriza por ataques de pánico recurrentes e inesperados, que son episodios de miedo intenso acompañados de síntomas físicos severos como palpitaciones, sudoración, temblores y sensación de asfixia. Las personas con trastorno de pánico a menudo viven con el miedo constante de que ocurran más ataques, lo que puede llevarlas a evitar situaciones o lugares donde han tenido ataques antes (Kessler et al., 2006).

Fobias Específicas: Las fobias son miedos intensos e irracionales hacia objetos o situaciones específicas, como alturas, animales o volar. Estos miedos desproporcionados pueden llevar a la evitación extrema de los estímulos fóbicos y pueden interferir significativamente en la vida cotidiana de una persona (APA, 2013).

Trastorno de Ansiedad Social: También conocido como fobia social, este trastorno implica un miedo intenso y persistente a ser juzgado, avergonzado o humillado en situaciones sociales o de desempeño. Las personas con trastorno de ansiedad social pueden evitar interacciones sociales o soportarlas con gran angustia (Stein \& Stein, 2008).

Trastorno Obsesivo-Compulsivo (TOC): Aunque el TOC se clasifica separadamente en el DSM-5, está estrechamente relacionado con los trastornos de ansiedad. Se caracteriza por pensamientos obsesivos intrusivos y comportamientos compulsivos repetitivos destinados a reducir la ansiedad causada por estas obsesiones (American Psychiatric Association, 2013).

Trastorno de Estrés Postraumático (TEPT): Este trastorno se desarrolla después de que una persona ha estado expuesta a un evento traumático. Los síntomas incluyen re-experimentación del trauma a través de flashbacks y pesadillas, evitación de recordatorios del evento, y aumento de la excitabilidad o reactividad (Yehuda, 2002).

Trastorno de Ansiedad por Separación: Es más común en niños, pero también puede afectar a adultos. Se caracteriza por un miedo excesivo a la separación de las personas a las que el individuo está apegado. Los síntomas pueden incluir angustia extrema cuando se anticipa o se experimenta la separación, así como preocupaciones persistentes acerca de que algo malo le sucederá a la persona apegada (APA, 2013).

\section{Trabajos relacionados}

Se decribirán en este apartado otros trabajos y desarrollos previos que hayan abordado cuestiones similares o relacionadas con los objetivos del trabajo actual.

