A lo largo del estado de la cuestión, será necesario abordar los campos relevantes de la psicología, la música y los videojuegos para fundamentar el desarrollo de la aplicación con investigaciones pertinentes. A continuación, se detallan los campos de interés que contribuirán a alcanzar los objetivos de la investigación.

\section{Marco teórico del trabajo}

\subsection{Musicoterapia y ansiedad}

\subsubsection{Definición de ansiedad}

La ansiedad es una emoción que todos experimentamos en menor o mayor medida en algunos momentos de nuestra vida. Es una respuesta natural del cuerpo ante situaciones de peligro o estrés y puede manifestarse de varias formas. La ansiedad se define generalmente como un sentimiento de miedo, preocupación o malestar relacionado con eventos futuros inciertos. Fisiológicamente, la ansiedad activa el sistema nervioso simpático, desencadenando respuestas corporales como aumento del ritmo cardíaco, sudoración y tensión muscular (\cite{APA:2013}). Emocionalmente, puede provocar sensaciones de temor, y cognitivamente, puede conducir a pensamientos intrusivos acerca de posibles peligros o fracasos.

Según la American Psychological Association (\citeyear{APA:2020}), la ansiedad es distinta al miedo, ya que esta última es una respuesta a una amenaza inmediata, mientras que la ansiedad implica la anticipación a una amenaza futura. La ansiedad es un estado emocional que prepara a la persona para enfrentar potenciales peligros. Esta respuesta puede ser adaptativa cuando se presenta en niveles moderados, ya que puede mejorar el rendimiento y la atención. Sin embargo, si la ansiedad es excesiva y persistente, puede interferir de manera significativa en la vida diaria de una persona. En este último caso, sería crucial consultar a un especialista para identificar el grado de ansiedad que sufre la persona y tratarla de manera adecuada.

Los síntomas de la ansiedad se pueden clasificar en tres dimensiones: físicas, emocionales y cognitivas. Los síntomas físicos incluyen palpitaciones, sudoración, temblores, tensión muscular, dificultad para respirar, mareos, fatiga y problemas gastrointestinales. Son el resultado de la activación del sistema nervioso simpático, que prepara al cuerpo para la lucha o huida en situaciones percibidas como peligrosas (\cite{CRASKE:2016}). A nivel emocional, las personas con ansiedad a menudo experimentan una sensación constante de miedo o preocupación. Es posible que se sientan inquietos, irritables y que perciban una pérdida de control sobre sus emociones (\cite{BARLOW:2002}). En relación a los síntomas cognitivos, la ansiedad puede alterar la forma en que las personas piensan, provocando pensamientos intrusivos, preocupaciones excesivas y dificultades para concentrarse. Estos síntomas cognitivos pueden perpetuar el ciclo de ansiedad. La preocupación constante sobre posibles peligros puede intensificar los síntomas físicos y emocionales (\cite{CLARK:2011}). Todos estos síntomas pueden presentarse simultáneamente, complicando la vida cotidiana de quienes padecen ansiedad. Los niveles de ansiedad están directamente relacionados con la tolerancia de la persona y las circunstancias que rodean la aparición de estos síntomas.

\subsubsection{Tipos de trastornos de ansiedad}
Existen varios tipos de trastornos de ansiedad, cada uno con sus características y criterios diagnósticos propios. Los trastornos de ansiedad más comunes incluyen:

\begin{itemize}
	\item \textbf{Trastorno de Ansiedad Generalizada (TAG):} se caracteriza por una preocupación excesiva e incontrolable en relación con diversas actividades o eventos. Esto se acompaña de síntomas físicos como tensión muscular y problemas de sueño. Las personas con este trastorno suelen anticipar desastres y están constantemente preocupadas por su salud, dinero, familia o trabajo (\cite{APA:2013}).
	\item \textbf{Trastorno de pánico:} este trastorno se caracteriza por ataques de pánico recurrentes e inesperados. Estos son episodios de intenso miedo acompañados de síntomas físicos severos, que incluyen palpitaciones, sudoración, temblores y sensación de asfixia. Las personas con trastorno de pánico a menudo viven con el temor constante de sufrir más ataques. Esto puede llevarlas a evitar situaciones o lugares donde han experimentado ataques anteriormente (\cite{KESSLER:2005}).
	\item \textbf{Trastorno de ansiedad social:} también conocido como fobia social, este trastorno implica un temor intenso y persistente a ser juzgado, avergonzado o humillado en situaciones sociales o de desempeño. Las personas con trastorno de ansiedad social pueden evitar las interacciones sociales o soportarlas con gran angustia (\cite{STEIN:2008}).
	\item \textbf{Trastorno Obsesivo-Compulsivo (TOC):} aunque el TOC se clasifica de manera independiente en el DSM-5\footnote{El DSM-5, actualizado en 2013, es el Manual Diagnóstico y Estadístico de los Trastornos Mentales. Su objetivo es asistir a los profesionales de la salud en el diagnóstico de los trastornos mentales de los pacientes.}, está estrechamente relacionado con los trastornos de ansiedad. Este trastorno se caracteriza por pensamientos obsesivos intrusivos y comportamientos compulsivos repetitivos, los cuales están destinados a reducir la ansiedad causada por estas obsesiones (\cite{APA:2013}).
	\item \textbf{Trastorno de Estrés Postraumático (TEPT):} este trastorno se desarrolla después de que una persona ha sido expuesta a un evento traumático. Los síntomas incluyen la experimentación del trauma a través de flashbacks y pesadillas, la evitación de los recordatorios del evento, y un aumento de la excitabilidad o reactividad (\cite{YEHUDA:2002}).
	\item \textbf{Trastorno de ansiedad por separación:} es más frecuente en niños, pero también puede afectar a adultos. Se caracteriza por un miedo excesivo a la separación de las personas a las que el individuo está apegado. Los síntomas pueden incluir una angustia extrema al anticipar o experimentar la separación, así como preocupaciones constantes de que algo malo suceda a la persona a la que está apegado (\cite{APA:2013}).
	\item \textbf{Fobias específicas:} las fobias son miedos intensos e irracionales hacia objetos o situaciones específicas, como por ejemplo, las alturas, los animales o volar. Estos miedos exagerados pueden provocar una evitación extrema de los estímulos que los causan e interferir significativamente en la vida diaria de una persona (\cite{APA:2013}).
\end{itemize}

\subsubsection{Terapias tradicionales para la ansiedad}

Para tratar la ansiedad se emplean diversas terapias tradicionales, las cuales han demostrado ser efectivas para las afecciones físicas y psicológicas de los pacientes. Estas se pueden clasificar en terapias psicológicas y tratamientos farmacológicos. 

Las terapias psicológicas son intervenciones respaldadas por evidencia que se enfocan en cambiar los patrones de pensamiento y comportamiento que contribuyen a la ansiedad. Entre las más efectivas están la terapia cognitivo-conductual (TCC), la terapia de exposición, y la terapia de aceptación y compromiso (ACT).

La TCC es ampliamente reconocida como una de las terapias más efectivas para tratar la ansiedad. Esta se centra en identificar y transformar patrones de pensamiento negativos y comportamientos disfuncionales que alimentan la ansiedad. Los pacientes aprenden técnicas de afrontamiento y estrategias de relajación para manejar su ansiedad de manera más efectiva (\cite{HOFMANN:2012}).

La terapia de exposición, como sugiere su nombre, implica exponer al paciente gradualmente a los estímulos que le causan ansiedad en un entorno controlado. El objetivo disminuir la sensibilidad del paciente, reduciendo su respuesta de ansiedad. La exposición puede ser en vivo, donde el paciente enfrenta directamente la situación temida, o imaginaria, donde visualiza la situación (\cite{CRASKE:2008}). Esta terapia es particularmente eficaz en el tratamiento de fobias específicas, trastorno de pánico y TEPT (\cite{POWERS:2010}).

En cuanto a la terapia ACT, ésta se enfoca en aceptar pensamientos y sentimientos difíciles en lugar de resistirse a ellos. Además, promueve el compromiso con acciones que estén en línea con los valores personales del paciente. Esta terapia ayuda a los pacientes a llevar una vida significativa a pesar de la ansiedad, fomentando la flexibilidad psicológica (\cite{HAYES:2012}).

En relación con los tratamientos farmacológicos, estos ayudan a manejar los síntomas de la ansiedad. Los antidepresivos, particularmente los inhibidores selectivos de la recaptación de serotonina (ISRS) y los inhibidores de la recaptación de serotonina y norepinefrina (IRSN), son comúnmente usados para tratar la ansiedad. Estos medicamentos incrementan los niveles de neurotransmisores en el cerebro, lo cual mejora el estado de ánimo y reduce la ansiedad (\cite{BALDWIN:2014}). Por otro lado, los ansiolíticos como las benzodiazepinas proporcionan alivio rápido de los síntomas de ansiedad. No obstante, debido a su potencial de adicción y abuso, su uso generalmente se limita a corto plazo (\cite{FOND:2023}). Los betabloqueantes, como el propranolol, se utilizan para tratar los síntomas físicos de la ansiedad, como taquicardia y temblores. Estos medicamentos bloquean los efectos de la adrenalina, ayudando a reducir los síntomas físicos (\cite{STEENEN:2016}).

\subsubsection{Definición de musicoterapia}

La World Federation of Music Therapy (WFMT) define la musicoterapia de la siguiente manera: 

\begin{adjustwidth}{100pt}{0pt}
\textit{La musicoterapia se refiere al uso profesional de la música y sus elementos como una intervención en entornos médicos, educativos y cotidianos con individuos, grupos, familias o comunidades que buscan optimizar su calidad de vida y mejorar su salud física, social, comunicativa, emocional, intelectual y espiritual. La musicoterapia se utiliza para satisfacer las necesidades físicas, emocionales, cognitivas y sociales de los pacientes}. \\
\begin{flushright}
	\vspace{-30px}
	(\cite{WFMT:2024})
\end{flushright}
\end{adjustwidth}

La musicoterapia se basa en la premisa de que la música puede tener un impacto profundo en el cerebro y el cuerpo, facilitando procesos de curación y crecimiento personal. La terapia puede involucrar tanto la creación activa de música (canto, tocar instrumentos) como la escucha pasiva de música seleccionada cuidadosamente para inducir estados emocionales específicos. Existen diversas modalidades de musicoterapia, cada una con sus propios métodos y aplicaciones. Las dos categorías principales son la musicoterapia activa y la musicoterapia receptiva.

La musicoterapia activa implica la participación directa del paciente en la creación de música. Esto puede incluir tocar instrumentos, cantar, improvisar y componer música. La interacción activa con la música permite a los pacientes expresarse de maneras que pueden no ser posibles a través del lenguaje verbal, proporcionando una vía poderosa para la autoexpresión y el procesamiento emocional.

Tocar Instrumentos:

Descripción: Los pacientes utilizan una variedad de instrumentos musicales, como tambores, pianos y guitarras, para crear sonidos y ritmos. Esta actividad puede ayudar a mejorar la coordinación motora, la atención y la comunicación (Darnley-Smith \& Patey, 2003).
Evidencia de Eficacia: Un estudio de Gold et al. (2009) mostró que la musicoterapia activa puede reducir significativamente los síntomas de depresión y ansiedad en pacientes con enfermedades mentales. La participación activa en la creación musical también puede mejorar la autoestima y el sentido de logro.
Canto:

Descripción: Cantar puede ser una forma poderosa de expresión emocional. Los pacientes pueden cantar solos, en dúo con el terapeuta o en grupos. Esta actividad puede fortalecer la función respiratoria y la articulación del habla, además de fomentar la conexión social (Baker \& Uhlig, 2011).
Evidencia de Eficacia: Según Clift et al. (2010), cantar en un grupo puede tener efectos positivos en la salud mental y el bienestar, promoviendo sentimientos de pertenencia y apoyo social.
Improvisación:

Descripción: La improvisación permite a los pacientes crear música espontáneamente, sin restricciones ni juicios. Esta forma de expresión libre puede ayudar a los pacientes a explorar y procesar emociones complejas, facilitando el autoconocimiento y la liberación emocional (Bruscia, 1987).
Evidencia de Eficacia: Un meta-análisis de Geretsegger et al. (2014) indicó que la improvisación en musicoterapia puede ser efectiva en la mejora de la comunicación y las habilidades interpersonales en personas con trastornos del espectro autista.
Musicoterapia Receptiva
La musicoterapia receptiva, por otro lado, se centra en la escucha pasiva de música seleccionada por el terapeuta. Esta modalidad puede incluir la escucha de música en vivo o grabada, con el objetivo de inducir estados de relajación, evocación de recuerdos o el procesamiento de emociones.

Escucha de Música:

Descripción: Los pacientes escuchan música seleccionada por el terapeuta para inducir estados específicos de ánimo o para facilitar la reflexión y la conversación sobre experiencias emocionales. Esta técnica puede ser útil para la relajación y la reducción del estrés (Bruscia, 1998).
Evidencia de Eficacia: Un estudio de Pelletier (2004) encontró que la música puede reducir significativamente los niveles de estrés y ansiedad en pacientes preoperatorios. La música calmante puede disminuir la presión arterial, la frecuencia cardíaca y los niveles de cortisol.
Relajación Guiada con Música:

Descripción: En esta modalidad, el terapeuta guía al paciente a través de ejercicios de relajación mientras escucha música suave y calmante. Esta técnica puede ayudar a los pacientes a alcanzar un estado de relajación profunda, mejorando el sueño y reduciendo la tensión muscular (Burns et al., 2001).
Evidencia de Eficacia: Un estudio de Chang et al. (2015) mostró que la relajación guiada con música puede mejorar la calidad del sueño y reducir los niveles de ansiedad en pacientes con trastornos del sueño.
Evidencia Científica
La efectividad de la musicoterapia está respaldada por una creciente cantidad de investigación científica. Diversos estudios han demostrado que la musicoterapia puede ser una intervención efectiva para una amplia gama de condiciones, incluyendo trastornos mentales, enfermedades crónicas y condiciones neurológicas.

Trastornos Mentales: La musicoterapia ha demostrado ser efectiva en la reducción de síntomas de depresión, ansiedad y estrés en pacientes con trastornos mentales (Aalbers et al., 2017). Un meta-análisis de Maratos et al. (2008) encontró que la musicoterapia puede ser tan efectiva como las terapias psicológicas tradicionales para la depresión.

Enfermedades Crónicas: En pacientes con enfermedades crónicas como el cáncer, la musicoterapia puede mejorar la calidad de vida, aliviar el dolor y reducir la ansiedad (Bradt et al., 2016). Un estudio de Lin et al. (2011) indicó que la musicoterapia puede mejorar el bienestar emocional y físico de los pacientes con cáncer en tratamiento.

Condiciones Neurológicas: La musicoterapia puede mejorar las habilidades motoras, la comunicación y la función cognitiva en pacientes con condiciones neurológicas como el accidente cerebrovascular y el Parkinson (Magee et al., 2017). Un estudio de Sarkamo et al. (2008) mostró que la escucha regular de música puede mejorar la recuperación cognitiva y emocional en pacientes con accidente cerebrovascular.}

\section{Trabajos relacionados}

Se decribirán en este apartado otros trabajos y desarrollos previos que hayan abordado cuestiones similares o relacionadas con los objetivos del trabajo actual.

