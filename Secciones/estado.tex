A lo largo del estado de la cuestión, será necesario abordar los campos relevantes de la psicología, la música y los videojuegos para fundamentar el desarrollo de la aplicación con investigaciones pertinentes. A continuación, se detallan los campos de interés que contribuirán a alcanzar los objetivos de la investigación.

\section{Marco teórico del trabajo}

\subsection{Ritmo, creación musical y emociones}

\subsubsection{Definición de ritmo}

El ritmo es un elemento intrínseco de la vida humana. Se manifiesta en la mayoría de las formas de arte, siendo de gran importancia en la música, la poesía y la danza. La definición de ritmo según \citeauthor{CONCEPTO:2021} (\citeyear{CONCEPTO:2021}) es la siguiente:

\begin{adjustwidth}{100pt}{0pt}
	''\textit{Se denomina ritmo a todo movimiento regular y recurrente, marcado por una serie de eventos opuestos o diferentes que se suceden en el tiempo. Dicho en otras palabras, el ritmo es un fluir del movimiento de naturaleza visual o sonora, cuyo orden interno puede percibirse e incluso reproducirse.}''
\end{adjustwidth}

La percepción del ritmo es totalmente subjetiva, pero para comprenderlo es esencial encontrar la clave que lo hace común para todos los individuos de la sociedad. Un ritmo siempre será un ritmo si los eventos que ocurren en el tiempo definen un patrón y un margen de repetición. Existen distintos tipos de ritmo en varias áreas artísticas. Sin embargo, nos centraremos en el ritmo musical, que es el más relevante para el campo psicológico de la musicoterapia. El ritmo musical se compone de varios elementos que indican su velocidad, intensidad y duración. El pulso, el tempo y la métrica son los elementos principales que definen un ritmo musical. Abordaremos individualmente cada uno de estos términos, indagando en profundidad que función tienen dentro de la música.

\begin{itemize}
	\item \textbf{Pulso:} es considerado como el latido del corazón de la música. Por definición, el pulso consiste en una serie de pulsaciones que se repiten de manera constante en una pieza musical, vinculadas directamente al movimiento del pie al escuchar una canción. El pulso puede tener diferentes velocidades, ya sean más rápidas o más lentas, pero siempre es constante a lo largo de la pieza musical. Sin embargo, existen excepciones si el compositor de la música decide acelerar o disminuir la velocidad intencionalmente (\cite{VIOLÍNZN:2024}).
	\item \textbf{Tempo:} proveniente del italiano ''tiempo'', el tempo se refiere a la velocidad de una composición. Metafóricamente, es similar a un reloj, que nos indica cuándo realizar ciertas acciones. El tempo permite a los músicos conocer el momento preciso en el que tocar cada sección de una pieza musical. Cuando los pulsos están distanciados equitativamente en el tiempo, la unidad de medida del tempo son BPM (\cite{MASTEREDBLOGS:2021}). El tempo puede ser constante, como en algunas canciones contemporáneas, o variable. Es común encontrar movimientos con distintos tempos en las piezas clásicas.
	\item \textbf{Métrica:} en el punto medio entre los dos términos anteriores, encontramos la métrica. Esta combina ambos elementos y define la estructura de una composición musical. La métrica se refiere a cómo se organizan los pulsos y el tempo en una pieza musical, agrupando los pulsos en unidades llamadas compases. En cada compás, algunos pulsos sonarán más fuertes y otros más suaves, lo cual se conoce como acentuación. La combinación de todos estos elementos proporciona un sentido rítmico a la música (\cite{COMPOMUSICAL:2024}).
\end{itemize}


\subsection{Musicoterapia y ansiedad}

\subsubsection{Definición de ansiedad}

La ansiedad es una emoción que todos experimentamos en menor o mayor medida en algunos momentos de nuestra vida. Es una respuesta natural del cuerpo ante situaciones de peligro o estrés y puede manifestarse de varias formas. La ansiedad se define generalmente como un sentimiento de miedo, preocupación o malestar relacionado con eventos futuros inciertos. Fisiológicamente, la ansiedad activa el sistema nervioso simpático, desencadenando respuestas corporales como aumento del ritmo cardíaco, sudoración y tensión muscular (\cite{APA:2013}). Emocionalmente, puede provocar sensaciones de temor, y cognitivamente, puede conducir a pensamientos intrusivos acerca de posibles peligros o fracasos.

Según la American Psychological Association (\citeyear{APA:2020}), la ansiedad es distinta al miedo, ya que esta última es una respuesta a una amenaza inmediata, mientras que la ansiedad implica la anticipación a una amenaza futura. La ansiedad es un estado emocional que prepara a la persona para enfrentar potenciales peligros. Esta respuesta puede ser adaptativa cuando se presenta en niveles moderados, ya que puede mejorar el rendimiento y la atención. Sin embargo, si la ansiedad es excesiva y persistente, puede interferir de manera significativa en la vida diaria de una persona. En este último caso, sería crucial consultar a un especialista para identificar el grado de ansiedad que sufre la persona y tratarla de manera adecuada.

Los síntomas de la ansiedad se pueden clasificar en tres dimensiones: físicas, emocionales y cognitivas. Los síntomas físicos incluyen palpitaciones, sudoración, temblores, tensión muscular, dificultad para respirar, mareos, fatiga y problemas gastrointestinales. Son el resultado de la activación del sistema nervioso simpático, que prepara al cuerpo para la lucha o huida en situaciones percibidas como peligrosas (\cite{CRASKE:2016}). A nivel emocional, las personas con ansiedad a menudo experimentan una sensación constante de miedo o preocupación. Es posible que se sientan inquietos, irritables y que perciban una pérdida de control sobre sus emociones (\cite{BARLOW:2002}). En relación a los síntomas cognitivos, la ansiedad puede alterar la forma en que las personas piensan, provocando pensamientos intrusivos, preocupaciones excesivas y dificultades para concentrarse. Estos síntomas cognitivos pueden perpetuar el ciclo de ansiedad. La preocupación constante sobre posibles peligros puede intensificar los síntomas físicos y emocionales (\cite{CLARK:2011}). Todos estos síntomas pueden presentarse simultáneamente, complicando la vida cotidiana de quienes padecen ansiedad. Los niveles de ansiedad están directamente relacionados con la tolerancia de la persona y las circunstancias que rodean la aparición de estos síntomas.

\subsubsection{Tipos de trastornos de ansiedad}
Existen varios tipos de trastornos de ansiedad, cada uno con sus características y criterios diagnósticos propios. Los trastornos de ansiedad más comunes incluyen:

\begin{itemize}
	\item \textbf{Trastorno de Ansiedad Generalizada (TAG):} se caracteriza por una preocupación excesiva e incontrolable en relación con diversas actividades o eventos. Esto se acompaña de síntomas físicos como tensión muscular y problemas de sueño. Las personas con este trastorno suelen anticipar desastres y están constantemente preocupadas por su salud, dinero, familia o trabajo (\cite{APA:2013}).
	\item \textbf{Trastorno de pánico:} este trastorno se caracteriza por ataques de pánico recurrentes e inesperados. Estos son episodios de intenso miedo acompañados de síntomas físicos severos, que incluyen palpitaciones, sudoración, temblores y sensación de asfixia. Las personas con trastorno de pánico a menudo viven con el temor constante de sufrir más ataques. Esto puede llevarlas a evitar situaciones o lugares donde han experimentado ataques anteriormente (\cite{KESSLER:2005}).
	\item \textbf{Trastorno de ansiedad social:} también conocido como fobia social, este trastorno implica un temor intenso y persistente a ser juzgado, avergonzado o humillado en situaciones sociales o de desempeño. Las personas con trastorno de ansiedad social pueden evitar las interacciones sociales o soportarlas con gran angustia (\cite{STEIN:2008}).
	\item \textbf{Trastorno Obsesivo-Compulsivo (TOC):} aunque el TOC se clasifica de manera independiente en el DSM-5\footnote{El DSM-5, actualizado en 2013, es el Manual Diagnóstico y Estadístico de los Trastornos Mentales. Su objetivo es asistir a los profesionales de la salud en el diagnóstico de los trastornos mentales de los pacientes.}, está estrechamente relacionado con los trastornos de ansiedad. Este trastorno se caracteriza por pensamientos obsesivos intrusivos y comportamientos compulsivos repetitivos, los cuales están destinados a reducir la ansiedad causada por estas obsesiones (\cite{APA:2013}).
	\item \textbf{Trastorno de Estrés Postraumático (TEPT):} este trastorno se desarrolla después de que una persona ha sido expuesta a un evento traumático. Los síntomas incluyen la experimentación del trauma a través de flashbacks y pesadillas, la evitación de los recordatorios del evento, y un aumento de la excitabilidad o reactividad (\cite{YEHUDA:2002}).
	\item \textbf{Trastorno de ansiedad por separación:} es más frecuente en niños, pero también puede afectar a adultos. Se caracteriza por un miedo excesivo a la separación de las personas a las que el individuo está apegado. Los síntomas pueden incluir una angustia extrema al anticipar o experimentar la separación, así como preocupaciones constantes de que algo malo suceda a la persona a la que está apegado (\cite{APA:2013}).
	\item \textbf{Fobias específicas:} las fobias son miedos intensos e irracionales hacia objetos o situaciones específicas, como por ejemplo, las alturas, los animales o volar. Estos miedos exagerados pueden provocar una evitación extrema de los estímulos que los causan e interferir significativamente en la vida diaria de una persona (\cite{APA:2013}).
\end{itemize}

A pesar de que los trastornos de ansiedad representan el mayor problema de salud mental en los EE. UU, la mayoría de las personas nunca buscan ayuda médica. Estos muestran una etiología compleja, donde se reconoce el componente genético y los factores estresantes debido a los acontecimientos de la vida. Para su diagnóstico, los médicos y psiquiatras se basan en los criterios clínicos establecidos por el Manual Diagnóstico
y Estadístico de los Trastornos Mentales (DSM-5) (\cite{DELGADO:2021}).

\subsubsection{Terapias tradicionales para la ansiedad}

Para tratar la ansiedad se emplean diversas terapias tradicionales, las cuales han demostrado ser efectivas para las afecciones físicas y psicológicas de los pacientes. Estas se pueden clasificar en terapias psicológicas y tratamientos farmacológicos. Para decidir qué tratamiento utilizar, es necesario realizar una evaluación previa para identificar el tipo de trastorno que tiene el paciente. Aunque ambas opciones se utilizan para tratar la ansiedad, las terapias se consideran tratamientos de primera línea. Estas se combinan para modular los patrones de pensamiento (\cite{DELGADO:2021}).

Las terapias psicológicas son intervenciones respaldadas por evidencia que se enfocan en cambiar los patrones de pensamiento y comportamiento que contribuyen a la ansiedad. Entre las más efectivas están la terapia cognitivo-conductual (TCC), la terapia de exposición, y la terapia de aceptación y compromiso (ACT). Estas terapias no sólo funcionan y se utilizan individualmente, sino que también ofrecen la posibilidad de ser utilizadas de manera complementaria. De esta forma, el terapeuta puede aprovechar las virtudes de cada una para obtener los resultados requeridos en los objetivos terapéuticos.

La TCC es ampliamente reconocida como una de las terapias más efectivas para tratar la ansiedad. Esta se centra en identificar y transformar patrones de pensamiento negativos y comportamientos disfuncionales que alimentan la ansiedad. Los pacientes aprenden técnicas de afrontamiento y estrategias de relajación para manejar su ansiedad de manera más efectiva (\cite{HOFMANN:2012}).

La terapia de exposición, como sugiere su nombre, implica exponer al paciente gradualmente a los estímulos que le causan 	ansiedad en un entorno controlado. El objetivo es disminuir la sensibilidad del paciente, reduciendo su respuesta de ansiedad. La exposición puede ser en vivo, donde el paciente enfrenta directamente la situación temida, o imaginaria, donde visualiza la situación (\cite{CRASKE:2008}). Esta terapia es particularmente eficaz en el tratamiento de fobias específicas, trastorno de pánico y TEPT (\cite{POWERS:2010}).

En cuanto a la terapia ACT, esta se enfoca en aceptar pensamientos y sentimientos difíciles en lugar de resistirse a ellos. Además, promueve el compromiso con acciones que estén en línea con los valores personales del paciente. Esta terapia ayuda a los pacientes a llevar una vida significativa a pesar de la ansiedad, fomentando la flexibilidad psicológica (\cite{HAYES:2012}).

En relación con los tratamientos farmacológicos, estos ayudan a manejar los síntomas de la ansiedad. Los antidepresivos, particularmente los inhibidores selectivos de la recaptación de serotonina (ISRS) y los inhibidores de la recaptación de serotonina y norepinefrina (IRSN), son comúnmente usados para tratar la ansiedad. Estos medicamentos incrementan los niveles de neurotransmisores en el cerebro, lo cual mejora el estado de ánimo y reduce la ansiedad (\cite{BALDWIN:2014}). Por otro lado, los ansiolíticos como las benzodiazepinas proporcionan alivio rápido de los síntomas de ansiedad. No obstante, debido a su potencial de adicción y abuso, su uso generalmente se limita a corto plazo (\cite{FOND:2023}). Los betabloqueantes, como el propranolol, se utilizan para tratar los síntomas físicos de la ansiedad, como taquicardia y temblores. Estos medicamentos bloquean los efectos de la adrenalina, ayudando a reducir los síntomas físicos (\cite{STEENEN:2016}).

\subsubsection{Definición de musicoterapia}

La World Federation of Music Therapy (WFMT) define la musicoterapia de la siguiente manera; citamos literalmente: 

\begin{adjustwidth}{100pt}{0pt}
\textit{''La musicoterapia se refiere al uso profesional de la música y sus elementos como una intervención en entornos médicos, educativos y cotidianos con individuos, grupos, familias o comunidades que buscan optimizar su calidad de vida y mejorar su salud física, social, comunicativa, emocional, intelectual y espiritual. La musicoterapia se utiliza para satisfacer las necesidades físicas, emocionales, cognitivas y sociales de los pacientes}.'' \\
\begin{flushright}
	\vspace{-30px}
	(\cite{WFMT:2024})
\end{flushright}
\end{adjustwidth}

La música se ha utilizado como herramienta terapéutica durante muchos años. Como medio de expresión no verbal, la música facilita la comunicación y la exteriorización de sentimientos, permitiendo a las personas explorar o reexplorar su interior y compartirlo con los demás. El objetivo de usar música es facilitar la expresión emocional del individuo y su desarrollo comunicativo (\cite{TRESIERRA:2005}). Es importante incorporar la música en la educación emocional, especialmente en los niños. Debemos inculcar la importancia de la música como una herramienta en sus vidas que les ayude a gestionar sus emociones. Según \citeauthor{POCH:2001} (\citeyear{POCH:2001}), la música no debe considerarse superflua, ya que ofrece una serie de aportaciones más allá de ser simplemente un pasatiempo. La música tiene un impacto inmediato en los seres humanos, afectando aspectos biológicos, físicos, neurológicos, psicológicos, sociales y espirituales. Además, la música es un patrón autocurativo que la humanidad siempre ha utilizado. Por ejemplo, para aliviar tensiones, cubrir carencias afectivas o expresar sentimientos de alegría a través de la danza, el dolor de una muerte o el amor en canciones románticas. La música acompaña a una persona en todos los momentos esenciales de su vida. Esta presencia se ve reforzada por la cultura, que ha creado canciones y composiciones para cada uno de estos momentos.

La musicoterapia tiene aplicaciones en diversos campos psicológicos. Por ejemplo, en la Unidad de Cuidados Intensivos (UCI) Pediátrica, \citeauthor{TRESIERRA:2005} (\citeyear{TRESIERRA:2005}) explica que la música actúa como un agente relajante, permitiendo al paciente comunicar sus emociones en un entorno donde se sienta escuchado. Además, el efecto relajante de la música está directamente relacionado con la percepción del dolor y el estrés, lo que puede tener un impacto positivo en el sistema inmunológico del paciente. Esta serie de reacciones en cadena mejora el bienestar biológico del paciente. Según \citeauthor{TRESIERRA:2005} (\citeyear{TRESIERRA:2005}), las terapias para niños autistas pueden beneficiarse de la musicoterapia. Esta puede romper el aislamiento social y mejorar el desarrollo socioemocional. En estos casos, un instrumento musical puede actuar como intermediario entre el paciente y el terapeuta, fortaleciendo su comunicación.

\subsubsection{Modalidades de musicoterapia}

La musicoterapia abarca diversos métodos y aplicaciones, agrupándose en dos modalidades distintas. La musicoterapia activa es aquella donde el paciente dirige la acción, combinándola normalmente con la creación musical como base fundamental. En contraste, la musicoterapia receptiva es aquella donde el paciente se convierte en un sujeto pasivo, como por ejemplo, escuchando cierto tipo de música relajante, y que impacta en sus emociones de manera inherente.

La musicoterapia activa conlleva que el paciente participe activamente en el proceso terapéutico, ya sea cantando, tocando un instrumento o bailando. En este escenario, el paciente se convierte en el intérprete o creador de la música, mientras que el terapeuta sirve como un acompañante del proceso. Un elemento crucial en la musicoterapia activa es la improvisación musical, la cual se define como el arte de crear música de manera espontánea al tocar. No se trata de crear música desde cero, sino más bien de responder a experiencias musicales previamente aprendidas. En estas terapias, el terapeuta no solo acompaña al paciente, sino que también puede participar activamente junto a él. Es fundamental que el terapeuta comprenda las intenciones creativas del paciente, ya que si no hay una conexión musical entre ambos, la musicoterapia activa no será efectiva (\cite{SALAMANCA:2003}).

En la musicoterapia receptiva o pasiva, dependiendo del contexto, a diferencia de la anterior modalidad, el paciente escucha música grabada o interpretada en vivo por el terapeuta. Aunque el paciente sigue participando en la sesión, su función cambia, convirtiéndose en un agente pasivo. Este tipo de musicoterapia puede aliviar el estrés y la ansiedad, mejorar la concentración y la creatividad, entre otros beneficios. Existen dos acercamientos comunes de la musicoterapia receptiva: el canto armónico y los cuencos tibetanos. El canto armónico es una hermosa forma de expresión que consiste en cantar dos, tres o incluso cuatro sonidos simultáneamente, utilizando la mayor cantidad posible de resonadores dentro del cuerpo y el cráneo. Por otro lado, los cuencos tibetanos están compuestos por hasta siete metales diferentes\footnote{Según las tradiciones, cada metal está relacionado con un astro diferente de nuestro sistema solar (\cite{SALAMANCA:2003}).}, que producen distintas resonancias armónicas. Estos cuencos, que se utilizan comúnmente para la meditación, se llenan con distintos niveles de agua para conseguir diferentes tipos de sonidos y efectos cuando se golpean o se frotan (\cite{SALAMANCA:2003}).

Como explica \citeauthor{BRUSCIA:1998-2} (\citeyear{BRUSCIA:1998-2}), en relación con ambas modalidades, cada sesión de musicoterapia involucra al paciente en algún tipo de experiencia musical. Estos diferentes tipos de experiencias musicales pueden incorporarse a las terapias de manera única o complementaria, según lo decida el terapeuta en función de las necesidades del paciente. Los cuatro tipos básicos son:

\begin{itemize}
	\item \textbf{Improvisación:} el paciente crea música ya sea cantando o tocando un instrumento de forma espontánea.
	\item \textbf{Recreación:} el paciente canta o toca una pieza musical preexistente, ya sea de memoria o con partitura.
	\item \textbf{Composición:} con la ayuda del terapeuta, el objetivo es componer y anotar una pieza musical en una partitura.
	\item \textbf{Escuchar:} el cliente escucha música, ya sea pregrabada o en vivo.
\end{itemize}

La improvisación, la recreación y la composición forman parte de la musicoterapia activa. Cada uno de estos enfoques ofrece una serie de beneficios únicos. Es útil implementar en la terapia cada uno de ellos de manera específica después de analizar el tipo de patología al que va dirigido. Sin embargo, la escucha es parte de la musicoterapia receptiva.

La música actúa como aliada en la terapia, colaborando con el terapeuta para intervenir conjuntamente. El terapeuta puede utilizar la música para alcanzar sus objetivos, ya sea por sí misma o en combinación con intervenciones personales. La naturaleza de las intervenciones musicales en la terapia puede variar según la estética que se desee transmitir al paciente. Sonido, belleza y creatividad se unen en la musicoterapia para ofrecer una metodología poderosa en la gestión emocional de los pacientes (\cite{BRUSCIA:1998-2}). Además, la musicoterapia es una opción poderosa en las terapias de grupo. Como explican \citeauthor{PRIETO:2022} (\citeyear{PRIETO:2022}), la música une a las personas sin importar la nacionalidad, el color o la raza. Contribuye a nuestra construcción y abre un camino de trascendencia más allá de nuestro cuerpo e intelecto. Independientemente de nuestra condición física o psicológica, la evocación emocional de la música es común a todas las personas.

\subsubsection{Beneficios de la musicoterapia en el tratamiento de la ansiedad}

La musicoterapia ha surgido como un tratamiento eficaz para gestionar la ansiedad, fundamentada por una creciente base de evidencia científica. Varios estudios y ensayos clínicos han demostrado que la musicoterapia puede disminuir significativamente los niveles de ansiedad en poblaciones de distintos contextos, desde pacientes hospitalizados hasta individuos con trastornos de ansiedad generalizados. Un estudio realizado por \citeauthor{SEPULVEDA:2014} (\citeyear{SEPULVEDA:2014}) en una población de niños de entre 8 y 16 años con diferentes tipos de cáncer, encontró una disminución significativa en los niveles de ansiedad tanto antes (prequimioterapia) como después (posquimioterapia) de la quimioterapia. No obstante, cuando se combinó esta terapia convencional con la escucha de música, la reducción en los niveles de ansiedad fue sustancialmente mayor. Para ser precisos, los niveles de ansiedad pasaron a reducirse del 27\% al 95\%.

\section{Trabajos relacionados}

Se decribirán en este apartado otros trabajos y desarrollos previos que hayan abordado cuestiones similares o relacionadas con los objetivos del trabajo actual.

