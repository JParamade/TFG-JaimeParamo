A lo largo del estado de la cuestión, será necesario abordar los campos relevantes de la psicología, la música y los videojuegos para fundamentar el desarrollo de la aplicación con investigaciones pertinentes. A continuación, se detallan los campos de interés que contribuirán a alcanzar los objetivos de la investigación.

\section{Marco teórico del trabajo}

\subsection{Musicoterapia y ansiedad}

\subsubsection{Definición de ansiedad}

La ansiedad es una emoción humana común que todos experimentamos en algunos momentos de nuestra vida. Es una respuesta natural del cuerpo ante situaciones de peligro o estrés y puede manifestarse de varias formas. Para entender mejor la ansiedad, es importante considerar tanto su definición como los diferentes tipos de trastornos de ansiedad.

La ansiedad se define generalmente como un sentimiento de miedo, preocupación o malestar relacionado con eventos futuros inciertos. Fisiológicamente, la ansiedad activa el sistema nervioso simpático, desencadenando respuestas corporales como aumento del ritmo cardíaco, sudoración y tensión muscular (\cite{APA:2013}). Emocionalmente, puede provocar sensaciones de temor, y cognitivamente, puede conducir a pensamientos intrusivos acerca de posibles peligros o fracasos.

Según la American Psychological Association (\citeyear{APA:2020}), la ansiedad es distinta al miedo, ya que esta última es una respuesta a una amenaza inmediata, mientras que la ansiedad implica la anticipación a una amenaza futura. La ansiedad es un estado emocional que prepara a la persona para enfrentar potenciales peligros. Esta respuesta puede ser adaptativa cuando se presenta en niveles moderados, ya que puede mejorar el rendimiento y la atención. Sin embargo, si la ansiedad es excesiva y persistente, puede interferir de manera significativa en la vida diaria de una persona.

Los síntomas de la ansiedad se pueden clasificar en tres dimensiones: físicas, emocionales y cognitivas. Los síntomas físicos incluyen palpitaciones, sudoración, temblores, tensión muscular, dificultad para respirar, mareos, fatiga y problemas gastrointestinales. Son el resultado de la activación del sistema nervioso simpático, que prepara al cuerpo para la lucha o huida en situaciones percibidas como peligrosas (\cite{CRASKE:2016}). A nivel emocional, las personas con ansiedad a menudo experimentan una sensación constante de miedo o preocupación. Es posible que se sientan inquietos, irritables y que perciban una pérdida de control sobre sus emociones (\cite{BARLOW:2002}). En relación a los síntomas cognitivos, la ansiedad puede alterar la forma en que las personas piensan, provocando pensamientos intrusivos, preocupaciones excesivas y dificultades para concentrarse. Estos síntomas cognitivos pueden perpetuar el ciclo de ansiedad. La preocupación constante sobre posibles peligros puede intensificar los síntomas físicos y emocionales (\cite{CLARK:2011}). Todos estos síntomas pueden presentarse simultáneamente, complicando la vida cotidiana de quienes padecen ansiedad. Los niveles de ansiedad están directamente relacionados con la tolerancia de la persona y las circunstancias que rodean la aparición de estos síntomas.

\subsubsection{Tipos de trastornos de ansiedad}
Existen varios tipos de trastornos de ansiedad, cada uno con sus características y criterios diagnósticos propios. Los trastornos de ansiedad más comunes incluyen:

\begin{itemize}
	\item \textbf{Trastorno de Ansiedad Generalizada (TAG):} se caracteriza por una preocupación excesiva e incontrolable en relación con diversas actividades o eventos. Esto se acompaña de síntomas físicos como tensión muscular y problemas de sueño. Las personas con este trastorno suelen anticipar desastres y están constantemente preocupadas por su salud, dinero, familia o trabajo (\cite{APA:2013}).
	\item \textbf{Trastorno de pánico:} este trastorno se caracteriza por ataques de pánico recurrentes e inesperados. Estos son episodios de intenso miedo acompañados de síntomas físicos severos, que incluyen palpitaciones, sudoración, temblores y sensación de asfixia. Las personas con trastorno de pánico a menudo viven con el temor constante de sufrir más ataques. Esto puede llevarlas a evitar situaciones o lugares donde han experimentado ataques anteriormente (\cite{KESSLER:2005}).
	\item \textbf{Trastorno de ansiedad social:} también conocido como fobia social, este trastorno implica un temor intenso y persistente a ser juzgado, avergonzado o humillado en situaciones sociales o de desempeño. Las personas con trastorno de ansiedad social pueden evitar las interacciones sociales o soportarlas con gran angustia (\cite{STEIN:2008}).
	\item \textbf{Trastorno Obsesivo-Compulsivo (TOC):} aunque el TOC se clasifica de manera independiente en el DSM-5\footnote{El DSM-5, actualizado en 2013, es el Manual Diagnóstico y Estadístico de los Trastornos Mentales. Su objetivo es asistir a los profesionales de la salud en el diagnóstico de los trastornos mentales de los pacientes.}, está estrechamente relacionado con los trastornos de ansiedad. Este trastorno se caracteriza por pensamientos obsesivos intrusivos y comportamientos compulsivos repetitivos, los cuales están destinados a reducir la ansiedad causada por estas obsesiones (\cite{APA:2013}).
	\item \textbf{Trastorno de Estrés Postraumático (TEPT):} este trastorno se desarrolla después de que una persona ha sido expuesta a un evento traumático. Los síntomas incluyen la experimentación del trauma a través de flashbacks y pesadillas, la evitación de los recordatorios del evento, y un aumento de la excitabilidad o reactividad (\cite{YEHUDA:2002}).
	\item \textbf{Trastorno de ansiedad por separación:} es más frecuente en niños, pero también puede afectar a adultos. Se caracteriza por un miedo excesivo a la separación de las personas a las que el individuo está apegado. Los síntomas pueden incluir una angustia extrema al anticipar o experimentar la separación, así como preocupaciones constantes de que algo malo suceda a la persona a la que está apegado (\cite{APA:2013}).
	\item \textbf{Fobias específicas:} las fobias son miedos intensos e irracionales hacia objetos o situaciones específicas, como las alturas, los animales o volar. Estos miedos exagerados pueden provocar una evitación extrema de los estímulos que los causan e interferir significativamente en la vida diaria de una persona (\cite{APA:2013}).
\end{itemize}





\section{Trabajos relacionados}

Se decribirán en este apartado otros trabajos y desarrollos previos que hayan abordado cuestiones similares o relacionadas con los objetivos del trabajo actual.

