The aim of the ARTEMIS project is to develop an application that serves as a complementary tool in music therapy sessions that aim to help patients move from a state of anxiety to one of calm. To achieve this goal, research has been carried out on the functioning of anxiety, its possible treatments, music therapy therapies and their applications in the relevant psychological field, as well as the design elements of serious health-oriented video games. An application has been developed that encompasses various activities with common goals, using technologies that enable the creation of interactive digital experiences such as Unity, audio management such as FMOD, and collaborative work in repositories such as GitHub. Specifically, the activity that has been developed in this work immerses the patient in a relaxing rhythmic experience in which they will have to find a subjective solution to a musical puzzle. The elaboration of this experience following the therapeutic bases of relaxation and the interactivity of the user-centered digital medium, allows the therapist to adapt to the needs of each patient, and even to the different needs of the same patient at different stages of therapy.

\textit{Keywords:} video games, music therapy, rhythm, mental health, anxiety, ARTEMIS