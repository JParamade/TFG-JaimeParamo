\section{Conclusión}

La ansiedad es una emoción común en la vida humana, presente desde la niñez hasta la vejez. Es probable que tengamos que lidiar con sus síntomas en algún momento de nuestras vidas. El proyecto ARTEMIS se creó para proporcionar soporte digital a las terapias psicológicas enfocadas en el tratamiento de las emociones con connotaciones negativas mediante la musicoterapia. Para esta etapa del proyecto en concreto, la investigación se centró alrededor del tratamiento de la ansiedad. Al inicio de este TFG, se fijaron unos objetivos generales y específicos que se alinearon con los del proyecto ARTEMIS. El objetivo principal se ha centrado en desarrollar una aplicación con enfoque de videojuego serio, basándose en los principios del proyecto. Los objetivos secundarios, en cambio, han incluido el estudio de estructuras musicales para identificar patrones que los pacientes puedan integrar en sus creaciones musicales, así como el análisis del impacto de las experiencias interactivas digitales en las prácticas tradicionales de musicoterapia.

A lo largo del trabajo se han cumplido los distintos objetivos en sus respectivas secciones. Tanto el estudio de las estructuras musicales como el análisis del impacto de las experiencias interactivas digitales en musicoterapia se han realizado en el marco teórico. Con el conocimiento de las figuras rítmicas y el proceso de composición, se pueden crear piezas musicales que relajen al paciente. Además, si la pieza es creada por el propio paciente, la satisfacción en su creación es subjetiva, facilitando el proceso de mejora. Las experiencias interactivas digitales no solo aumentan la interacción entre el terapeuta y el paciente, sino que también sitúan a ambos en una situación de doble usuario donde pueden cooperar para alcanzar el objetivo principal de la terapia. Es importante destacar que la adaptabilidad de este formato permite al terapeuta ajustar las terapias fácilmente según las necesidades del paciente. Lograr cumplir ambos objetivos desemboca en el desarrollo de la aplicación que no solo debe ser funcional desde un punto de vista terapéutico, sino que también debe tener la capacidad de establecer una conexión emocional entre el terapeuta y el paciente para facilitar una comunicación efectiva.

En rasgos generales, la investigación se enfocó en los pilares fundamentales del proyecto ARTEMIS: las emociones (específicamente la ansiedad), el ritmo y la creación musical, la musicoterapia y el desarrollo de videojuegos serios. Todos estos temas, que abarcan las áreas de conocimiento de la psicología, la música y los videojuegos, se exploraron a fondo en el marco teórico. En primer lugar, se definieron el ritmo y sus componentes principales (pulso, tempo y métrica) utilizando diversas fuentes. Citamos a \citeauthor{CASTELLANOS:2009} (\citeyear{CASTELLANOS:2009}) y su explicación de cómo el ritmo juega un papel esencial en la música, permitiendo marcar el tiempo de una pieza musical. Es similar a cómo miramos el reloj para realizar ciertas acciones. Además, el ritmo, siendo un movimiento cíclico, tiene una gran relación con la inhalación y exhalación de la respiración. Este aspecto confiere al ritmo un potencial terapéutico de grandes dimensiones.

Se exploró también el proceso de composición musical y las diversas maneras de componer música. Utilizamos el trabajo de \citeauthor{SUBIRATS:2004} (\citeyear{SUBIRATS:2004}) para investigar los diferentes niveles de creatividad que otorgan diversas habilidades en la creación. Es importante destacar la diferencia entre la improvisación libre y la composición guiada. Ambos tipos de creación son útiles en entornos terapéuticos y, combinados con las diferentes formas de generar música, brindan a los terapeutas un abanico de opciones para sus sesiones. Sin embargo, es necesario evaluar las necesidades del paciente para determinar qué tipo de creación musical se debe utilizar. En relación con el oyente de música, existe un término conocido como expectativa musical que juega un papel crucial en el manejo de las emociones. \citeauthor{SENABRE:2019} (\citeyear{SENABRE:2019}) expone una serie de teorías que fundamentan la expectativa musical. Estas teorías nos ayudan a comprender qué espera un paciente de su creación musical y cómo podemos ayudarle a alcanzar un resultado satisfactorio.

Para comprender la situación de un paciente con síntomas de ansiedad, fue necesario definir esta emoción, los tipos de trastornos que existen y sus tratamientos. Aunque nuestro caso de estudio está enfocado en las terapias psicológicas para tratar la ansiedad, también ampliamos la información para incluir tratamientos farmacológicos con el mismo objetivo. Se examinó la relación entre esta emoción y la musicoterapia. Antes de hacerlo, definimos qué es la musicoterapia. Nos basamos en \citeauthor{WFMT:2024} (\citeyear{WFMT:2024}), una federación especializada en musicoterapia, para comprender el verdadero significado de las terapias basadas en música. Observamos que la musicoterapia se divide en dos modalidades: activa y receptiva. En la musicoterapia activa, el paciente se encuentra en un contexto donde él es el creador de la música, mientras que en la musicoterapia receptiva, el paciente escucha música grabada o interpretada en vivo por el terapeuta. La elección entre una u otra modalidad recae en el terapeuta, quien debe evaluar las circunstancias del paciente para tomar tal decisión. Se ha demostrado que la música es un componente poderoso en el tratamiento de la ansiedad. Un estudio realizado por \citeauthor{SEPULVEDA:2014} (\citeyear{SEPULVEDA:2014}) mostró que la musicoterapia reducía drásticamente los niveles de ansiedad en pacientes infantiles con cáncer.

Finalmente, se exploraron los elementos de diseño que definen al videojuego serio, enfatizando las características que lo distinguen del formato tradicional del videojuego. Se especificó en los diferentes campos de aplicación de los videojuegos serios, como la educación o la salud, proporcionando ejemplos de desarrollos específicos cuyo objetivo era otorgar una mejora a los usuarios en cada campo particular. La investigación culminó con una recopilación de videojuegos serios para la salud, detallando sus objetivos, los elementos de diseño que incluyen y cómo podrían ser útiles en nuestro desarrollo. \textit{Operation Quest} (\cite{OPERATIONQUEST:2024}), por su enfoque hacia el público infantil; \textit{SPARX} (\cite{SPARX:2013}), por su narrativa que explica al paciente su situación; \textit{Flow} (\cite{FLOW:2006}), por su simplicidad mecánica; y \textit{Deep VR} (\cite{DEEP:2021}), por su ambiente inmersivo e innovadoras aplicaciones de interacción, proporcionan una perspectiva amplia de los objetivos del proyecto ARTEMIS.

Se decidió adoptar una metodología que imita los procesos de desarrollo de un videojuego tradicional. Las características principales en la creación de un videojuego incluyen la definición de requisitos antes de comenzar, la conceptualización y diseño de las mecánicas del juego, y el desarrollo iterativo. En nuestro caso específico, para ofrecer un diseño de juego sólido, es necesario fundamentar todas las decisiones de diseño en las investigaciones realizadas previamente. Hemos utilizado tecnologías que facilitan la creación de experiencias digitales interactivas, como Unity, la gestión de audio a través de FMOD, y el trabajo colaborativo en repositorios como GitHub. En combinación con la metodología estipulada, estas herramientas nos han permitido desarrollar una aplicación interactiva de musicoterapia, sin requerir que el equipo esté en el mismo lugar o tiempo. Aplicando esta metodología, se logró desarrollar un sistema interactivo en el que el paciente puede disfrutar de la creación musical como si resolviera un rompecabezas sonoro. La solución se encuentra en la satisfacción personal del paciente.

En conclusión, analizar el impacto de las experiencias digitales interactivas en las sesiones de musicoterapia mostró que estas no solo proporcionan un medio adaptativo que permite al terapeuta personalizar las terapias según las necesidades del paciente, sino que también sumergen al paciente en un entorno visual y sonoro artístico que puede ser más efectivo que las terapias tradicionales. Este formato permite al terapeuta tener un registro en tiempo real de la interacción del paciente con la aplicación, lo que posibilita el análisis de las necesidades del paciente al instante. Para el usuario, los efectos de la terapia son inmediatos debido a la interacción instantánea. En cuanto al desarrollo de la aplicación, que se basa en la modalidad de musicoterapia activa, se fundamenta en la investigación realizada. Combina las contribuciones de diferentes referencias para crear un puzle sonoro relajante en el que los pacientes pueden disfrutar creando música y asienta las bases de una posible ampliación futura hacia un alcance de emociones más allá de la ansiedad.

Este Trabajo Final de Grado no tiene pruebas empíricas que demuestren la utilización de esta aplicación en situaciones con pacientes reales; se fundamenta únicamente en las investigaciones realizadas. Aunque la intención inicial era utilizar la aplicación en terapias reales llevadas a cabo por los psicólogos asociados al proyecto ARTEMIS, la falta de tiempo impidió que esto ocurriera en esta primera etapa del proyecto. Haber podido comprobar la eficacia y recibir retroalimentación de los pacientes infantiles que sufren síntomas de algún tipo de ansiedad, nos habría permitido crear un producto final más iterado y pulido.

\section{Líneas de investigación futuras}

Aunque el proyecto ARTEMIS se ha centrado inicialmente en la ansiedad, tiene como objetivo expandir gradualmente su enfoque para tratar otras emociones a través de la musicoterapia. El desarrollo de esta aplicación establece las bases para un sistema complementario a las sesiones de terapia tradicional del terapeuta. El uso de un medio digital interactivo para el desarrollo ofrece un comportamiento modular que permite una gran escalabilidad del proyecto. Con el arte y las líneas narrativas ya definidos, la investigación de emociones adicionales solo requerirá identificar los problemas asociados con esas emociones y cómo tratarlos. El contexto artístico y narrativo ya está establecido.